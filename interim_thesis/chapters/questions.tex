\chapter{Research questions}\label{ch:questions}
\section{Main research question}
The main research question in this work is as follows:
\begin{researchq} \label{rq:main}
    \par
    How can we sharpen the CG iteration bound for Schwarz-preconditioned high-contrast heterogeneous scalar-elliptic problems beyond the classical condition number-based bound?
\end{researchq}
For instance, in \cite{ams_coarse_space_comp_study_Alves2024}, the AMS and GDSW preconditioners significantly outperform the RGDSW preconditioner, despite all three having similar condition numbers. The key differences appear in their spectral gap and cluster width, highlighting the need for additional spectral characteristics to refine existing bounds.

\section{Subsidiary research questions}
\begin{subsidiaryq} \label{rq:subsidiaries}
    To answer the main research question, we address the following subsidiary questions:
    \setlength\itemindent{1in}
    \begin{enumerate}[label=\textbf{Q\arabic*}, ref=\textbf{Q\arabic*}, leftmargin=1cm]
        \item\label{rq:subsidiary:measures} What spectral characteristics, like the condition number, can we define to estimate the distribution of eigenvalues in the eigenspectrum in the case of high-contrast heterogeneous problems?
        \item\label{rq:subsidiary:estimation} How can we estimate any of the spectral characteristics defined in \ref{rq:subsidiary:measures} for the eigenspectrum in the particular case of a model Darcy problem?
        \item\label{rq:subsidiary:heuristic} Given a certain eigenspectrum, how can we sharpen the CG iteration bound?
        \item\label{rq:subsidiary:performance} How does the sharpened bound from \ref{rq:subsidiary:heuristic} perform for an unpreconditioned Darcy problem in comparison with the classical bound in \cref{eq:cg_convergence_rate}?
        \item\label{rq:subsidiary:dependance} How does the performance described in \ref{rq:subsidiary:performance} of a sharpened bound vary with the measures found in \ref{rq:subsidiary:measures}?
        \item\label{rq:subsidiary:preconditioners} How can we employ the sharpened bound to distinguish between the performance of Schwarz-like preconditioners? 
        % \item\label{rq:subsidiary:formilization} Can we formalize a method to estimate the eigenspectrum beyond the condition number?
    \end{enumerate}
\end{subsidiaryq}

\section{Motivation}
This research is important because current studies primarily focus on selecting between different Schwarz preconditioners for the Darcy problem, yet condition numbers fail to distinguish them effectively. The ability to differentiate preconditioners based on spectral properties would improve the selection process. Applying sharpened bounds could lead to better predictions of preconditioner performance and improved efficiency in solving high-contrast problems.

\section{Challenges}\label{sec:challenges}
The main challenge lies in quantitatively estimating the spectrum of the preconditioned system. The literature provides a priori condition number estimates for various Schwarz preconditioners. For instance, in the simple cases of the additive Schwarz preconditioner with either a \ref{ASM_coarse_space:nicolaides} or \ref{ASM_coarse_space:local_eigenfunctions} an a priori estimate for the condition number is given by \cref{eq:two_level_ASM_condition_number} in combination with either \cref{eq:c0_nicolaides} or \cref{eq:c0_local_eigenfunctions}, respectively. The same can be said for the MsFEM and ACMS preconditioners. Despite this, there is no established method for estimating the full eigenspectrum. Without such an estimate, refining the CG iteration bound remains difficult. Overcoming this limitation is central to this work.