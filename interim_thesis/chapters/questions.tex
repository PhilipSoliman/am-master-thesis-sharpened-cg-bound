\chapter{Research questions}\label{ch:questions}\newpage
% \todo{Restate main research question – Clearly defined again.}\newline
% How can we sharpen the CG iteration bound for Schwarz preconditioned heterogeneous elliptic problems beyond the classical condition number based bound?
% \todo{Subsidiary questions – Justify why each one is important.}\newline
% \begin{enumerate}
%     \item How does the performance of the sharpened bounds vary with cluster width and eigengap?
%     \item Can we come up with more numbers like the condition number that can be used to estimate the eigenspectrum?
%     \item How do the improved bounds compare with existing bounds in the literature?
%     \item How do the improved bounds perform in the context of high-contrast Darcy problems?
%     \item How to estimate eigenspectrum beyond the condition number?
% \end{enumerate}

% \todo{Connection to literature gaps – Explain why your questions matter.}\newline
% \begin{enumerate}
%     \item Current work focuses on how to choose between several Schwarz preconditioners for the Darcy problem. The condition number alone does not help to a priori distinguish between these preconditioners. For instance In \cite{ams_coarse_space_comp_study_Alves2024} the AMS and GDSW preconditioners perform extraordinarily well compared to the RGDSW preconditioner. However, all three preconditioned systems share similar condition number. The only difference is in their spectral gap and cluster width. This motivates the need for a new number that can help distinguish between these preconditioners.
%     \item Application of sharpened bounds to distinguish between the performance of Schwarz-like preconditioners
%     \item Focus on application of the improved bounds to the Darcy problem in the context of Schwarz like preconditioners
% \end{enumerate}

% \todo{Expected challenges – Possible difficulties in answering them.}\newline
% One of the biggest challenges in sharpening the CG iteration bound lies in the quantitative estimation of the preconditioned system's spectrum. The literature on condition number estimation for Schwarz preconditioners is vast. For instance, in the simple cases of the additive Schwarz preconditioner with either a \cref{ASM_coarse_space:nicolaides} or \cref{ASM_coarse_space:local_eigenfunctions} an a priori estimate for the condition number is given by \cref{eq:two_level_ASM_condition_number} in combination with either \cref{eq:c0_nicolaides} or \cref{eq:c0_local_eigenfunctions}, respectively. The same can be said for the MsFEM and ACMS preconditioners. However, the literature lacks a quantitative estimate for the spectrum of the preconditioned system. This is crucial for sharpening the CG iteration bound. The lack of such an estimate is the main challenge in this work.

The main research question in this work is as follows:
\begin{researchq} \label{rq:main}
    \par
    How can we sharpen the CG iteration bound for Schwarz-preconditioned high-contrast heterogeneous elliptic problems beyond the classical condition number-based bound?
\end{researchq}
For instance, in \cite{ams_coarse_space_comp_study_Alves2024}, the AMS and GDSW preconditioners significantly outperform the RGDSW preconditioner, despite all three having similar condition numbers. The key differences appear in their spectral gap and cluster width, highlighting the need for additional spectral characteristics to refine existing bounds.

\begin{subsidiaryq} \label{rq:subsidiaries}
    To answer the main research question, we address the following subsidiary questions:
    \setlength\itemindent{1in}
    \begin{enumerate}[label=\textbf{Q\arabic*}, ref=Q\arabic*, leftmargin=1cm]
        \item\label{rq:subsidiary:measures} Which numerical measures, like the condition number, can we define to estimate the distribution of eigenvalues in the eigenspectrum in the case of high-contrast heterogeneous problems?
        \item\label{rq:subsidiary:estimation} How can we estimate any of the numerical measures defined in \ref{rq:subsidiary:measures} for the eigenspectrum in the particular case of a model Darcy problem?
        \item\label{rq:subsidiary:heuristic} Given a certain eigenspectrum, how can we sharpen the CG iteration bound?
        \item\label{rq:subsidiary:performance} How does the sharpened bound from \ref{rq:subsidiary:heuristic} perform for an unpreconditioned Darcy problem in comparison with the classical bound in \cref{eq:cg_convergence_rate}?
        \item\label{rq:subsidiary:dependance} How does the performance described in \ref{rq:subsidiary:performance} of a sharpened bound vary with the measures found in \ref{rq:subsidiary:measures}?
        \item\label{rq:subsidiary:preconditioners} Can we employ the sharpened bound to distinguish between the performance of Schwarz-like preconditioners? 
        \item\label{rq:subsidiary:formilization} Can we formalize a method to estimate the eigenspectrum beyond the condition number?
    \end{enumerate}
\end{subsidiaryq}

This research is important because current studies primarily focus on selecting between different Schwarz preconditioners for the Darcy problem, yet condition numbers fail to distinguish them effectively. The ability to differentiate preconditioners based on spectral properties would improve the selection process. Applying sharpened bounds could lead to better predictions of preconditioner performance and improved efficiency in solving high-contrast problems.

The main challenge lies in quantitatively estimating the spectrum of the preconditioned system. The literature provides a priori condition number estimates for various Schwarz preconditioners. For instance, in the simple cases of the additive Schwarz preconditioner with either a \cref{ASM_coarse_space:nicolaides} or \cref{ASM_coarse_space:local_eigenfunctions} an a priori estimate for the condition number is given by \cref{eq:two_level_ASM_condition_number} in combination with either \cref{eq:c0_nicolaides} or \cref{eq:c0_local_eigenfunctions}, respectively. The same can be said for the MsFEM and ACMS preconditioners. Despite this, there is no established method for estimating the full eigenspectrum. Without such an estimate, refining the CG iteration bound remains difficult. Overcoming this limitation is central to this work.