\chapter{Preliminary Results}\label{ch:preliminary_results}\newpage
\todo{Description of the results – What did you compute or observe?}\newline
The preliminary results are obtained through a numerical implementation of the ideas discussed in \cite[Section 4]{cg_sharpened_convrate_Axelsson1976}. Therein \citeauthor{cg_sharpened_convrate_Axelsson1976} presents a sharpened CG iteration bound for two particular eigenspectra, which are described below.

On the eigenspectrum of $A$, consider two intervals $[a, b]$ and $[c, d]$ with $a < b < c < d$ such that all eigenvalues of $A$ are contained in the union of these two intervals. Additionally, we have $\kappa(A) = \frac{d}{a}$. We distinguish the following two cases:
\begin{equation}
    \sigma_1(A) = [a,b] \bigcup [c,d]
    \label{eq:two_clusters}
\end{equation}
\begin{equation}
    \sigma_2(A) = [c,d] \bigcup_{i, \lambda_i \in [a,b]} \lambda_i
    \label{eq:one_cluster_with_tail}
\end{equation}
\begin{figure}[H]
    \centering
    \includegraphics[width=\textwidth]{effect_of_eigenvalue_distribution_sharpened_bounds.pdf}
    \caption{Similar to \cref{fig:cg_effect_of_eigenvalue_distribution}, but with the sharpened CG bound $m_s$.}
    \label{fig:cg_sharpened_bound}
\end{figure}
\todo{Methods used – How did you obtain these results?}\newline
\todo{Comparison with expectations – How do the results align with existing theories?}\newline
\todo{Limitations of preliminary findings – What needs further refinement?}\newline
\todo{Implications for research questions – Do the results answer or refine the questions?}\newline