\chapter{Preliminary Results}\label{ch:preliminary_results}
The results described in this chapter are adapted from the ideas discussed in \cite[Section 4]{cg_sharpened_convrate_Axelsson1976}. Therein \citeauthor{cg_sharpened_convrate_Axelsson1976} presents a sharpened CG iteration bound for two particular eigenspectra, which are described in \cref{sec:cg_sharpened_convrate}. The sharpened bound is then generalized to multiple clusters in \cref{sec:multiple_clusters}. The results are then compared with the classical CG iteration bound in \cref{sec:cg_sharpened_convrate_numerical_experiments}. Finally, the implications of the sharpened bound for the research questions are discussed in \cref{sec:cg_sharpened_convrate_implications}.

\section{Two cluster case}\label{sec:cg_sharpened_convrate}
On the eigenspectrum of $A$, consider two intervals $[a, b]$ and $[c, d]$ with $0 < a < b < c < d$ such that all eigenvalues of $A$ are contained in the union of these two intervals. Additionally, we have $\kappa(A) = \frac{d}{a}$. We treat the following two cases simultaneously
\begin{equation}
    \sigma_1(A) = [a,b] \bigcup [c,d]
    \label{eq:two_clusters}
\end{equation}

\begin{equation}
    \sigma_2(A) = [c,d] \bigcup_{\substack{i=1 \\ \lambda_i \in [a,b]}}^{N_{\text{tail}}} \lambda_i
    \label{eq:one_cluster_with_tail}
\end{equation}
where $N_{\text{tail}}$ is the number of eigenvalues in the tail. The first case is a two-cluster eigenspectrum, while the second case has one cluster and a tail of eigenvalues.
 
The CG error \cref{eq:cg_error_bound} suggests we look for a polynomial $r_{\bar{m}}$ of degree $\bar{m}$ that satisfies the constraints of the minimization problem in order to find a iteration bound $m < \bar{m}$. In other words, we do not solve the minimization problem directly, but we make a clever selection of the polynomial $r_{\bar{m}}$ that satisfies the constraints. As a consequence, the actual minimizing polynomial might require a lower degree $m$ to satisfy the same relative error tolerance $\epsilon$.

\citeauthor{cg_sharpened_convrate_Axelsson1976} suggests we use not one monolithic residual polynomial function, but a multiplication of two residual polynomial functions $\hat{r}^{(i)}_p(x)$ and $\hat{r}_{\bar{m}-p}(x)$ for the two clusters. The superscript $^{(i)}$ corresponds to the two eigenspectra described above. The residual polynomial functions are defined using \cref{def:scaled_chebyshev_polynomial} and $\gamma = 0$ as follows
\begin{equation}
    \hat{r}^{(i)}_p (x)
    \begin{cases}
        \hat{C}_p&, \text{if } i = 1\\
        \overset{p}{\underset{i=1}{\prod}} (1 - x/\lambda_i)&, \text{if } i = 2, p = N_{\text{tail}}\\
    \end{cases}
    \label{eq:residual_polynomial_rm}
\end{equation}
and
\begin{equation}
    \hat{r}_{{\bar{m}}-p} (x) = \frac{C_{\bar{m}-p} \left(\frac{d + c - 2x}{d - c}\right)}{C_{\bar{m}-p}\left(\frac{d + c}{d - c}\right)},
    \label{eq:residual_polynomial_rpm}
\end{equation}
Indeed, the product $r_{\bar{m}} = \hat{r}_p \hat{r}_{\bar{m}-p} \in \mathcal{P}_{\bar{m}}$. Hence, we can use the residual polynomial functions to bound the error at the $m^{\text{th}}$ iterate. Now, we obtain the following intermediate bounds
\begin{subequations}
    \begin{align}
        \max_{\lambda \in [a,b]} |r_{\bar{m}}(\lambda)| \leq \max_{\lambda \in [a,b]} |\hat{r}^{(i)}_p(\lambda)| \max_{\lambda \in [a,b]} |\hat{r}_{\bar{m}-p}(\lambda)| &\leq \max_{\lambda \in [a,b]} |\hat{r}^{(i)}_p(\lambda)|, \ \text{and} \label{eq:residual_polynomial_bound_ab}\\
        \max_{\lambda \in [c,d]} |r_{\bar{m}}(\lambda)| \leq \max_{\lambda \in [c,d]} |\hat{r}^{(i)}_p(\lambda)| \max_{\lambda \in [c,d]} |\hat{r}_{\bar{m}-p}(\lambda)| &\leq \max_{\lambda \in [c,d]} |\hat{r}_{p}(\lambda)|/C_{\bar{m}-p}\left(\frac{d+c}{d-c}\right) \label{eq:residual_polynomial_bound_cd}
    \end{align}
\end{subequations}
where the first result follows from the fact that $|\hat{r}_{m-p}(x)| < 1 \ \forall x \in [a,b]$ and the second result from 
\[
    \left|C_{m-p}\left(\frac{d+c -2x}{d-c}\right)\right| < 1 \ \forall x \in [c,d].
\]

Furthermore, using the equality \cref{eq:chebyshev_polynomial_approximation}, we have
\begin{equation}
    \frac{1}{C_{k}\left(\frac{z_1 + z_2}{z_1 - z_2}\right)} \leq 2 \left(\frac{\sqrt{z_2} - \sqrt{z_1}}{\sqrt{z_2} + \sqrt{z_1}}\right)^k, \text{ for } z_1 > z_2 > 0 \text{ and } k \in \mathbb{N}^+,
    \label{eq:chebyshev_polynomial_bound}
\end{equation}
and
\begin{equation}
    \max_{\lambda \in [a,b]} |\hat{r}^{(i)}_p(\lambda)| \leq
    \begin{cases}
        2\left(\frac{\sqrt{b}-\sqrt{a}}{\sqrt{b}+\sqrt{a}}\right)^p=\eta_1 &, \text{if } i = 1,\\
        \left(\frac{b}{a}-1\right)^p=\eta_2 &, \text{if } i = 2, p = N_{\text{tail}},
    \end{cases}
    \label{eq:residual_polynomial_bound_ab_i}
\end{equation}
Note that if $i=1$ we can determine $p$ by requiring that the maximum of the residual polynomial function $\hat{r}^{(i)}_p$ in $[a,b]$ is equal to $\epsilon$. This gives the following equation
\begin{equation}
    p = \left\lceil\frac{1}{2}\sqrt{\frac{b}{a}}\ln{\epsilon} + 1\right\rceil
    \label{eq:chebyshev_degree_p}
\end{equation}
Also note that for $i=2$ $\hat{r}^{(2)}_p(\lambda) = 0 < \epsilon$ for all eigenvalues $\lambda \in [a,b]$.

Next, $\hat{r}^{(i)}_p$ in $[c,d]$ is bounded by its maximum value within $[a,b]$ multiplied by the polynomial that is the fastest growing polynomial in $\mathcal{P}_{p}$ outside- and bounded below 1 within $[a,b]$. This polynomial is again the (transformed) Chebyshev polynomial $C_{p}\left(\frac{2x - b - a}{b - a}\right)$. Therefore,
\begin{equation*}
    \max_{\lambda \in [c,d]} |\hat{r}^{(i)}_p(\lambda)| \leq \eta_i C_{p}\left(\frac{2d - b - a}{b + a}\right),
\end{equation*}
with $\eta_i$ as defined in \cref{eq:residual_polynomial_bound_ab_i}.

At this point we have ensured that $\max_{\lambda \in [a,b]}|r_{\bar{m}}|$ is bounded by $\epsilon$ using \cref{eq:residual_polynomial_bound_ab}. So it remains to bound $\max_{\lambda \in [c,d]}|r_{\bar{m}}|$ in \cref{eq:residual_polynomial_bound_cd}. Using above results we can write 
\begin{equation*}
    \max_{\lambda \in [c,d]} |r_{\bar{m}}(\lambda)| < \epsilon,
\end{equation*}
if we require that
\begin{equation}
    \eta_i C_{p}\left(\frac{2d - b - a}{b - a}\right) /C_{\bar{m}-p}\left(\frac{d+c}{d-c}\right) < \epsilon.
    \label{eq:relative_error_bound_mp}
\end{equation}
Using that for $x_1, x_2, x_3 \in \mathbb{R}^+$ with $x_1 > x_3$ and $z = \frac{x_1 - x_2}{x_3}$
\begin{align*}
    C_p(z) & \leq \left(z + \sqrt{z^2 - 1}\right)^p \\
    & = \left( \frac{x_1 - x_2}{x_3} + \sqrt{ \left[\frac{x_1 - x_2}{x_3}\right]^2 -1}\right)^p \\
    & \leq \left( \frac{x_1}{x_3} + \sqrt{ \left[\frac{x_1}{x_3}\right]^2 - 1}\right)^p \\
    & \leq \left( \frac{2x_1}{x_3}\right)^p,
\end{align*}
and substituting $x_1 = 2d$, $x_2 = b + a$ and $x_3 = b - a$ we obtain the following inequality
\begin{equation}
    \eta_i \left(\frac{4d}{b-a} \right)^p /C_{\bar{m}-p}\left(\frac{d+c}{d-c}\right) < \epsilon. 
    \label{eq:chebyshev_degree_p_bound}
\end{equation}
Moreover,
\begin{align*}
    \eta_i \left(\frac{4d}{b-a}\right)^p &= 
    \begin{cases}
        2\left(\frac{\sqrt{b} - \sqrt{a}}{\sqrt{b} + \sqrt{a}} \frac{4d}{b-a}\right)^p &, \text{if } i = 1\\
        \left(\frac{b - a}{a}\frac{4d}{b-a}\right)^p &, \text{if } i = 2,
    \end{cases}\\
    &=
    \begin{cases}
        2\left(\frac{4d}{b + 2\sqrt{ab} + a}\right)^p &, \text{if } i = 1\\
        \left(\frac{4d}{a}\right)^p &, \text{if } i = 2,
    \end{cases}\\
    &\leq 2
    \begin{cases}
        \left(\frac{4d}{b}\right)^p &, \text{if } i = 1\\
        \left(\frac{4d}{a}\right)^p &, \text{if } i = 2,
    \end{cases}
\end{align*}
We can therefore require that the bound in \cref{eq:chebyshev_degree_p_bound} is satisfied if we have
\[
    1/C_{\bar{m}-p}\left(\frac{d+c}{d-c}\right) \leq \frac{\epsilon}{2\left( \frac{4d}{e_i}\right)^p},
\]
where 
\[
    e_i = \begin{cases}
        b &, \text{if } i = 1\\
        a &, \text{if } i = 2.\\
    \end{cases}
\]
Again using \cref{eq:chebyshev_polynomial_bound} and solving for the degree $\bar{m} - p$ we obtain
\[
    \bar{m} - p \geq \frac{1}{2}\sqrt{\frac{d}{c}}\left(\ln{\epsilon} + p \ln{\frac{4d}{e_i}}\right),
\]
which leads to the following bound for the number of iterations \cite[Equation 4.4]{cg_sharpened_convrate_Axelsson1976}
\begin{equation}
    \bar{m}=\left\lceil\frac{1}{2} \sqrt{\frac{d}{c}} \ln (2 / \epsilon)+\left(1+\frac{1}{2} \sqrt{\frac{d}{c}} \ln (4 d / e_i)\right) p\right\rceil,
    \label{eq:cg_iteration_bound_2_clusters}
\end{equation}
where 
\[
    1 \leq p \leq \min\left(\left\lceil\frac{1}{2}\sqrt{\frac{b}{a}}\ln{\epsilon} + 1\right\rceil, N_{\text{tail}}\right).
\]

\section{Generalization to multiple clusters}\label{sec:multiple_clusters}
At this point we assume that we are dealing with an eigenspectrum of the form $\sigma_1(A)$, i.e. we are only treating case 1. In \cref{sec:cg_sharpened_convrate_implications}, it is reasoned that this is indeed a very applicable case for \cref{prob:elliptic_problem_discretized}.

In this case, the technique outlined in \cref{sec:cg_sharpened_convrate} starts at the left most cluster $[a,b]$, finds the Chebyshev degree $p_1=p$ satisfying \cref{eq:chebyshev_degree_p}, moves to the neighboring cluster $[c,d]$ and finds the Chebyshev degree $p_2 = \bar{m} - p$ satisfying \cref{eq:relative_error_bound_mp}. Rewriting \cref{eq:relative_error_bound_mp} gives the following equation for $p_2$:
\begin{equation}
    \frac{1}{C_{p_2}\left(\frac{d+c}{d-c}\right)} \leq \frac{\epsilon}{{C}^{(1)}_{p_1}(d)} = \epsilon_2,
    \label{eq:chebyshev_degree_p_prime}
\end{equation}
where
\[
    C^{(1)}_{p_1}(x) = C_{p_1}\left(\frac{b + a - 2x}{b - a}\right) /C_{p_1}\left(\frac{b+a}{b-a}\right),
\]
is the Chebyshev polynomial corresponding to the first cluster.

Suppose there is a third cluster to the right of $[c,d]$, i.e. $[e,f]$. We can repeat the process and find the Chebyshev degree $p_3$ satisfying a similar as \cref{eq:chebyshev_degree_p_prime} for the third cluster. 
\[
    \frac{1}{C_{p_3}\left(\frac{f+e}{f-e}\right)} \leq \frac{\epsilon}{C^{(1)}_{p_1}(f)C^{(2)}_{p_2}(f)} = \epsilon_3,
\]
This leads to the general equation for the Chebyshev degree $p_i$ of the $i^{\text{th}}$ cluster $[a_i, b_i]$
\begin{equation}
    \frac{1}{C_{p_i}\left(\frac{b_i + a_i}{b_i - a_i}\right)} \leq \frac{\epsilon}{\prod_{j=1}^{i-1} C^{(j)}_{p_j}(b_i)} = \epsilon_i.
    \label{eq:chebyshev_degree_p_i}
\end{equation}

Due to the large range of the Chebyshev polynomials $\tilde{C}_p$ computer is likely to result in floating point number overflow during calculation of the denominator of \cref{eq:chebyshev_degree_p_i}. Instead, we first apply \cref{eq:chebyshev_polynomial_bound} and introduce the cluster condition numbers $\kappa_i = \frac{b_i}{a_i}$, where $i$ is the index of the cluster. We can then rewrite \cref{eq:chebyshev_degree_p_i} as follows
\begin{equation*}
    p_i  =  \left\lceil\ln{\frac{\epsilon_i}{2}} / \ln{\frac{\sqrt{\kappa_i} - 1}{\sqrt{\kappa_i} + 1}}\right\rceil, \\
\end{equation*}
and
\begin{align*}
    \ln{\frac{\epsilon_i}{2}} & = \ln{\frac{\epsilon}{2}} - \sum_{j=1}^{i-1} \ln{C^{(j)}_{p_j}(b_i)}. \\
\end{align*}
Let $z^{(i,j)}_1 = \frac{b_j + a_j - 2b_i}{b_j - a_j}$ and $z^{(j)}_2 = \frac{b_j + a_j}{b_j - a_j}$ then
\begin{equation*}
    \ln{C^{(j)}_{p_j}(b_i)} = \ln{C_{p_j}(z^{(i,j)}_1)} - \ln{C_{p_j}(z^{(j)}_2)}.
\end{equation*}
We have, using the definition of the Chebyshev polynomial
\begin{equation}
    \ln{C_{p_j}(z^{(i,j)}_1)} \lessapprox p_j \ln{\left[z^{(i,j)}_1 - \sqrt{\left(z^{(i,j)}_1\right)^2 - 1}\right]} - \ln{2},
    \label{eq:chebyshev_polynomial_bound_z1}
\end{equation}
and
\begin{equation}
    \ln{C_{p_j}(z^{(j)}_2)} \gtrapprox p_j \ln{\left[z^{(j)}_2 + \sqrt{\left(z^{(j)}_2\right)^2 - 1}\right]} - \ln{2},
    \label{eq:chebyshev_polynomial_bound_z2}
\end{equation}
both of which become more accurate approximations as $z,m\rightarrow\infty$. Introducing 
\begin{align*}
    \zeta^{(i,j)}_1 &= z^{(i,j)}_1 - \sqrt{\left(z^{(i,j)}_1\right)^2 - 1}, \\
    \zeta^{(j)}_2 &= z^{(j)}_2 + \sqrt{\left(z^{(j)}_2\right)^2 - 1}, \text{ and}\\
    f_i &= \frac{\sqrt{\kappa_i} - 1}{\sqrt{\kappa_i} + 1},
\end{align*}
with $\kappa_i$ the $i^{\text{th}}$ cluster condition number, and substituting the inequalities \ref{eq:chebyshev_polynomial_bound_z1} and \ref{eq:chebyshev_polynomial_bound_z2} back into the bound for $p_i$ gives
\begin{align*}
    p_i &\leq \left\lceil\frac{\ln{\frac{\epsilon}{2}} - \sum_{j=1}^{i-1} p_j\left(\ln{\zeta^{(i,j)}_1} - \ln{\zeta^{(j)}_2} \right)}{\ln{f_i}}\right\rceil \\
    &= \left\lceil\log_{f_i}{\frac{\epsilon}{2}} - \sum_{j=1}^{i-1} p_j\left(\log_{f_i}{\zeta^{(i,j)}_1} - \log_{f_i}{\zeta^{(j)}_2} \right)\right\rceil\\
    &= \left\lceil\log_{f_i}{\frac{\epsilon}{2}} - \sum_{j=1}^{i-1} p_j\log_{f_i}{\frac{\zeta^{(i,j)}_1}{\zeta^{(j)}_2}} \right\rceil
\end{align*}
Note that in general $\zeta^{(i,j)}_1 < \zeta^{(j)}_2$ and hence $\log_{f_i}{\left(\frac{\zeta^{(j)}_2}{\zeta^{(i,j)}_1}\right)} > 0$. This prompts us to write
\begin{equation}
    p_i \leq \left\lceil\log_{f_i}{\frac{\epsilon}{2}} + \sum_{j=1}^{i-1} p_j\log_{f_i}{\frac{\zeta^{(j)}_2}{\zeta^{(i,j)}_1}} \right\rceil
    \label{eq:chebyshev_degree_p_i_explicit}
\end{equation}
Evidently, adding more clusters to the left of the interval $[a_i,b_i]$ increases the degree $p_i$ of the Chebyshev polynomial. Next to this, \cref{eq:chebyshev_degree_p_i_explicit} reduces to the classical CG iteration bound \cref{eq:cg_convergence_rate_bound} for a single cluster when $i = N_{\text{clusters}} = 1$.

Equation \ref{eq:chebyshev_degree_p_i_explicit} gives us a way to calculate the Chebyshev degree $p_i$ of the $i^{\text{th}}$ cluster $[a_i,b_i]$ in terms of the Chebyshev degrees of the previous clusters. To obtain a bound on the number of iterations for the CG method we sum the Chebyshev degrees of all the clusters
\begin{equation}
    \bar{m} = \sum_{i=1}^{N_{\text{clusters}}} p_i
    \label{eq:cg_iteration_bound_multiple_clusters}
\end{equation}

\section{Numerical experiments}\label{sec:cg_sharpened_convrate_numerical_experiments}
From \cref{eq:chebyshev_degree_p_i_explicit} and \cref{eq:cg_iteration_bound_multiple_clusters} we get a sequential algorithm for determining an upper bound on the number of iterations for the CG method. A comparison of this new multiple-cluster bound with the classical CG bound from \cref{eq:cg_convergence_rate_bound_iterations} is shown in \cref{fig:cg_sharpened_bound}. As is the case for \cref{fig:cg_effect_of_eigenvalue_distribution}, $m_{\text{classical}} = 21$
\begin{figure}[H] 
    \centering
    \includegraphics[width=\textwidth]{effect_of_eigenvalue_distribution_sharpened_bounds.pdf}
    \caption{As in \cref{fig:cg_effect_of_eigenvalue_distribution}, plots of the last three CG residual polynomials for different eigenvalue distributions. $n_c$ indicates the number of clusters and $\sigma$ is the width of the cluster. The size of the system $N$ and the condition number $\kappa(A)$ are kept constant. $m$ indicates the number of iterations required for convergence. However, here $\bar{m}$ is determined by \cref{eq:chebyshev_degree_p_i_explicit,eq:cg_iteration_bound_multiple_clusters}.}
    \label{fig:cg_sharpened_bound}
\end{figure}
Figure \ref{fig:cg_sharpened_bound} shows that the sharpened CG iteration bound is significantly lower than the classical CG iteration bound for the two cluster case. The performance of the sharpened bound does decrease as the number of clusters increases, as is evident from \cref{eq:chebyshev_degree_p_i_explicit,eq:cg_iteration_bound_multiple_clusters}. Performance also decreases as clusters become wider. So much so, that the sharpened bound is worse than the classical bound for the three cluster case with spread $\sigma = 0.04$ and for the four cluster case.

Worsening performance for the sharpened bound with increased cluster width is expected. Focussing on the two cluster case, we rediscover the ratios $\frac{d}{c}$ in \cref{eq:cg_iteration_bound_2_clusters} as well as $\frac{a}{b}$ in the corresponding equation for $p$. These ratios grow with increasing cluster width.

\section{Implications for research}\label{sec:cg_sharpened_convrate_implications}
The preliminary results discussed in this chapter show that we can find both an a priori analytic two-cluster (\cref{eq:cg_iteration_bound_2_clusters}) and multiple-cluster (\cref{eq:cg_iteration_bound_multiple_clusters}) sharpened iteration bound for the CG method. The latter can also be implemented as a sequential, numerical algorithm that can be applied to artificially constructed spectra in \cref{fig:cg_sharpened_bound}. The sharpened bound appears to perform best in the two-cluster case, which has the most correspondence to the eigenspectrum of a typical (preconditioned) Darcy problem. Hence, \ref{rq:subsidiary:heuristic} is answered positively. 

With regard to \ref{rq:subsidiary:measures}, the cluster condition number $\kappa_i$ is introduced as a measure of the cluster width. This suggests that we can use the cluster condition number to distinguish between different preconditioners which is a promising result for \ref{rq:subsidiary:preconditioners}. Moreover, inspecting \cref{eq:cg_iteration_bound_2_clusters} more closely for the eigenspectrum in \cref{eq:two_clusters} (case $i=1$) reveals the presence of a sort of spectral gap $\frac{d}{b}$. This serves as yet another candidate for a potential set of spectral characteristics to estimate the distribution of eigenvalues in the eigenspectrum.

A logical next step is to more rigorously investigate how the sharpened bound depends on $\kappa_i$ as well as the spectral gap (\ref{rq:subsidiary:dependance}). Subsequently, we can simulate the eigenspectrum of a Darcy problem and compare the sharpened bound with the classical bound (\ref{rq:subsidiary:performance}). This will lead to a clear understanding of the performance of the sharpened bound in the main problem context of this thesis: heterogeneous scalar elliptic problems with high-contrast coefficient.

Furthermore, we can construct, discretize, and precondition a model Darcy problem with the methods outlined in \cref{sec:tailored_coarse_spaces}. Then, we apply both the sharpened bound and the CG method to the resulting systems, and investigate how sharp the new bound is (\ref{rq:subsidiary:preconditioners}).

The main challenge described in \cref{sec:challenges} still stands. More work is needed to be able to use the sharpened bound for spectra that are not known or artificially constructed beforehand. The results in this chapter suggest that the cluster condition number is a good candidate for a measure of the eigenspectrum. However, it is not yet clear how to estimate the cluster condition number for a general eigenspectrum. This is an important step towards answering \ref{rq:subsidiary:estimation}.