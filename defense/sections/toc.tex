\begin{frame}[label=toc]{Structure}
    \frametitle{Structure}
    \begin{itemize}
        \item CG method: under the hood
        \item Classical iteration bound
        \item Influence of eigenvalue distribution
        \item High-contrast coefficients: split eigenspectrum
        \item Two-cluster bound
        \item Multi-cluster bound
        \item Partitioning
        \item Results on sharpness
        \item Practical estimation of new bounds
    \end{itemize}
\end{frame}

% \begin{comment}
% Level 1: Discretisation & intro CG
%     - FEM discretization $\Omega\rightarrow\mathcal{T}$, $\nabla \rightarrow A$, $u\rightarrow\mathbf{u}$, $f\rightarrow\mathbf{b}$
%     - Linear system of equations $A\mathbf{u}=\mathbf{b}$
%     - CG takes in initial guess $\mathbf{u}_0$ and provides updated solution $\mathbf{u}_m$ after $m$ iterations.
%     - Classical bound $m_1$ (condition number $\kappa(A)$)
%     - Research showed that $m_1$ can be too pessimistic for high-contrast problems.
%     - Restate R.Q.: "How do we improve/sharpen the classical bound $m_1$?"
% Level 2: CG convergence in detail
%     - CG solution polynomial $q$, given as $\mathbf{u}_m = \mathbf{u}_0 + q(A)\mathbf{r}_0$,
%     - Spectrum $\sigma(A) = \{\lambda_1, \ldots, \lambda_n\}$
%     - CG residual polynomial $r(\lambda) = 1 - \lambda q(\lambda)$
%     - 
% \end{comment}