\begin{frame}[label=opening]{Opening}
    \frametitle{Opening}
    \framesubtitle{Motivation}
        \begin{itemize}
            \item<1-> Opening question: "What do X, Y, Z have in common?". (Show X, Y and Z with single picture slides and make sure to give short descriptions. Go quickly through the single-picture slides.)
            \item<2-> Answer: "They can all be modelled by a..." high-contrast diffusion problem for $u, u_D \in H^1(\Omega)$, $f\in L^2(\Omega)$ and $\mathcal{C}\in L^{\infty}(\Omega)$
                  \begin{equation*}
                      \begin{aligned}
                          -\nabla\cdot\left(\mathcal{C}\nabla u\right) & = f \quad \text{in } \Omega           \\
                          u                                            & = u_D \quad \text{on } \partial\Omega
                      \end{aligned}
                  \end{equation*}
            \item<3-> Understanding the problem: We need to find $u \in H^1(\Omega)$ such that the above equation holds.
            \item<4-> Briefly explain why discretization is necessary for PDEs, for non-experts.
            \item<5-> FEM discretization $\Omega\rightarrow\mathcal{T}$, $\nabla \rightarrow A$, $u\rightarrow\mathbf{u}$, $f\rightarrow\mathbf{b}$
            \item<6-> Linear system of equations $A\mathbf{u}=\mathbf{b}$
            \item<7-> We can compute approximations $\mathbf{u}_1, \mathbf{u}_2, \ldots$ efficiently using the CG method.
            \item<8-> Main R.Q.: "How can we improve existing estimates on the total number of necessary CG iterations?"
        \end{itemize}
\end{frame}