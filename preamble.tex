% problem
\newtheoremstyle{boldnewline}
  {\topsep}         % Space above
  {\topsep}         % Space below
  {}                % Body font
  {}                % Indent amount
  {\bfseries}       % Theorem head font
  {\newline}        % Punctuation after theorem head
  {1em}             % Space after theorem head
  {}                % Theorem head spec

\theoremstyle{boldnewline}
\newtheorem{problem}{Problem}

% theorem
\newtheoremstyle{customthm}
{}
{}
{\upshape}
{}
{\bfseries}
{.}
{.5em}
{}
\theoremstyle{customthm}
\newtheorem{theorem}{Theorem}[chapter]
\newtheorem{definition}{Definition}[chapter]

%custom problem environment
%todo: make box outline thinner
%todo: reduce alpha of color
%feedback from lisi: the conten should stand out more than the box itself
\newtcbtheorem[number within=chapter,crefname={Problem}{Problems}]{fancyprob}{Problem}%
{colback=white,colframe=chapter,fonttitle=\bfseries}{prob}
\newtcbtheorem[number within=section,crefname={Problem}{Problems}]{APPfancyprob}{Problem}%
{colback=white,colframe=chapter,fonttitle=\bfseries}{prob}

% custom theorem environment
\newtcbtheorem[number within=chapter,crefname={Theorem}{Theorems}]{fancyth}{Theorem}%
{colback=white,colframe=subsection,fonttitle=\bfseries}{th}
\newtcbtheorem[number within=section,crefname={Theorem}{Theorems}]{APPfancyth}{Theorem}%
{colback=white,colframe=subsection,fonttitle=\bfseries}{th}

% custom definition environment
\newtcbtheorem[number within=chapter,crefname={Definition}{Definitions}]{fancydef}{Definition}%
{colback=white,colframe=subsubsection,fonttitle=\bfseries}{def}
\newtcbtheorem[number within=section,crefname={Definition}{Definitions}]{APPfancydef}{Definition}%
{colback=white,colframe=subsubsection,fonttitle=\bfseries}{def}

% research question environments
\newtheorem*{researchq}{Research Question}
\newtheorem*{subsidiaryq}{Subsidiary Questions}

%tikz pictures
\usetikzlibrary{intersections, arrows.meta, calc}

% pgf figures
\usepackage{pgf}
\usepackage{pgfplots}
\pgfplotsset{compat=1.14}
% \usepackage{lmodern}
\usepackage{import}
\def\mathdefault#1{#1}
\everymath=\expandafter{\the\everymath\displaystyle}

% comment macros
\newif\ifshowcomments
\showcommentstrue
% \showcommentsfalse
\ifshowcomments
    \newcommand{\mynote}[2]{\fbox{\bfseries\sffamily\scriptsize{#1}}
        {\small$\blacktriangleright$\textsf{\emph{#2}}$\blacktriangleleft$}}
    \newcommand{\citehere}[0]{\textcolor{red}{\fbox{\bfseries\sffamily\scriptsize{CITATION}}}}
\else
    \newcommand{\mynote}[2]{}
    \newcommand{\citehere}[0]{}
\fi
\newcommand{\todo}[1]{\textcolor{blue}{\mynote{To do}{#1}}}

% subfiles (for chapters)
\usepackage{subfiles}

% subfile macros
\newcommand{\onlyinsubfile}[1]{#1}
\newcommand{\notinsubfile}[1]{}

% Bibliography
% \usepackage[backend=biber, style=apa, citestyle=ieee]{biblatex}
\usepackage[backend=biber,style=numeric,sortcites,sorting=nty,backref,natbib,hyperref]{biblatex}

% set list style
\setlist[enumerate, 1]{% global settings for the first level in enumerate environments
  leftmargin = \parindent, % item text indentation
  align = left,
  labelwidth=\parindent,
  labelsep = 2pt
}


\newcommand{\fullref}[1]{\cref{#1} \nameref{#1}}
