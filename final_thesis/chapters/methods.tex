\chapter{Methodology}\label{ch:methods}
The methods described in this chapter are adapted from the ideas discussed in \cite[Section 4]{cg_sharpened_convrate_Axelsson1976}. Therein \citeauthor{cg_sharpened_convrate_Axelsson1976} presents a sharpened CG iteration bound for two particular eigenspectra, which are described in \cref{sec:cg_sharpened_convrate}. The sharpened bound is then generalized to multiple clusters in \cref{sec:multiple_clusters}. The results are then compared with the classical CG iteration bound in \cref{sec:cg_sharpened_convrate_numerical_experiments}. Finally, the implications of the sharpened bound for the research questions are discussed in \cref{sec:cg_sharpened_convrate_implications}.

\section{Two cluster case}\label{sec:cg_sharpened_convrate}
On the eigenspectrum of $A$, consider two intervals $[a, b]$ and $[c, d]$ with $0 < a < b < c < d$ such that all eigenvalues of $A$ are contained in the union of these two intervals. Additionally, we have $\kappa(A) = \frac{d}{a}$. We treat the following two cases simultaneously
\begin{equation}
    \sigma_1(A) = [a,b] \bigcup [c,d]
    \label{eq:two_clusters}
\end{equation}

\begin{equation}
    \sigma_2(A) = [c,d] \bigcup_{\substack{i=1 \\ \lambda_i \in [a,b]}}^{N_{\text{tail}}} \lambda_i
    \label{eq:one_cluster_with_tail}
\end{equation}
where $N_{\text{tail}}$ is the number of eigenvalues in the tail. The first case is a two-cluster eigenspectrum, while the second case has one cluster and a tail of eigenvalues.
 
The CG error \cref{eq:cg_error_bound} suggests we look for a polynomial $r_{\bar{m}}$ of degree $\bar{m}$ that satisfies the constraints of the minimization problem in order to find a iteration bound $m < \bar{m}$. In other words, we do not solve the minimization problem directly, but we make a clever selection of the polynomial $r_{\bar{m}}$ that satisfies the constraints. As a consequence, the actual minimizing polynomial might require a lower degree $m$ to satisfy the same relative error tolerance $\epsilon$.

\citeauthor{cg_sharpened_convrate_Axelsson1976} suggests we use not one monolithic residual polynomial function, but a multiplication of two residual polynomial functions $\hat{r}^{(i)}_p(x)$ and $\hat{r}_{\bar{m}-p}(x)$ for the two clusters. The superscript $^{(i)}$ corresponds to the two eigenspectra described above. The residual polynomial functions are defined using \cref{def:scaled_chebyshev_polynomial} and $\gamma = 0$ as follows
\begin{equation}
    \hat{r}^{(i)}_p (x)
    \begin{cases}
        \hat{C}_p&, \text{if } i = 1\\
        \overset{p}{\underset{i=1}{\prod}} (1 - x/\lambda_i)&, \text{if } i = 2, p = N_{\text{tail}}\\
    \end{cases}
    \label{eq:residual_polynomial_rm}
\end{equation}
and
\begin{equation}
    \hat{r}_{{\bar{m}}-p} (x) = \frac{C_{\bar{m}-p} \left(\frac{d + c - 2x}{d - c}\right)}{C_{\bar{m}-p}\left(\frac{d + c}{d - c}\right)},
    \label{eq:residual_polynomial_rpm}
\end{equation}
Indeed, the product $r_{\bar{m}} = \hat{r}_p \hat{r}_{\bar{m}-p} \in \mathcal{P}_{\bar{m}}$. Hence, we can use the residual polynomial functions to bound the error at the $m^{\text{th}}$ iterate. Now, we obtain the following intermediate bounds
\begin{subequations}
    \begin{align}
        \max_{\lambda \in [a,b]} |r_{\bar{m}}(\lambda)| \leq \max_{\lambda \in [a,b]} |\hat{r}^{(i)}_p(\lambda)| \max_{\lambda \in [a,b]} |\hat{r}_{\bar{m}-p}(\lambda)| &\leq \max_{\lambda \in [a,b]} |\hat{r}^{(i)}_p(\lambda)|, \ \text{and} \label{eq:residual_polynomial_bound_ab}\\
        \max_{\lambda \in [c,d]} |r_{\bar{m}}(\lambda)| \leq \max_{\lambda \in [c,d]} |\hat{r}^{(i)}_p(\lambda)| \max_{\lambda \in [c,d]} |\hat{r}_{\bar{m}-p}(\lambda)| &\leq \max_{\lambda \in [c,d]} |\hat{r}_{p}(\lambda)|/C_{\bar{m}-p}\left(\frac{d+c}{d-c}\right) \label{eq:residual_polynomial_bound_cd}
    \end{align}
\end{subequations}
where the first result follows from the fact that $|\hat{r}_{m-p}(x)| < 1 \ \forall x \in [a,b]$ and the second result from 
\[
    \left|C_{m-p}\left(\frac{d+c -2x}{d-c}\right)\right| < 1 \ \forall x \in [c,d].
\]

Furthermore, using the equality \cref{eq:chebyshev_polynomial_approximation}, we have
\begin{equation}
    \frac{1}{C_{k}\left(\frac{z_1 + z_2}{z_1 - z_2}\right)} \leq 2 \left(\frac{\sqrt{z_2} - \sqrt{z_1}}{\sqrt{z_2} + \sqrt{z_1}}\right)^k, \text{ for } z_1 > z_2 > 0 \text{ and } k \in \mathbb{N}^+,
    \label{eq:chebyshev_polynomial_bound}
\end{equation}
and
\begin{equation}
    \max_{\lambda \in [a,b]} |\hat{r}^{(i)}_p(\lambda)| \leq
    \begin{cases}
        2\left(\frac{\sqrt{b}-\sqrt{a}}{\sqrt{b}+\sqrt{a}}\right)^p=\eta_1 &, \text{if } i = 1,\\
        \left(\frac{b}{a}-1\right)^p=\eta_2 &, \text{if } i = 2, p = N_{\text{tail}},
    \end{cases}
    \label{eq:residual_polynomial_bound_ab_i}
\end{equation}
Note that if $i=1$ we can determine $p$ by requiring that the maximum of the residual polynomial function $\hat{r}^{(i)}_p$ in $[a,b]$ is equal to $\epsilon$. This gives the following equation
\begin{equation}
    p \geq \left\lfloor\frac{1}{2}\sqrt{\frac{b}{a}}\ln{\frac{2}{\epsilon}} + 1\right\rfloor
    \label{eq:chebyshev_degree_p}
\end{equation}
Also note that for $i=2$ $\hat{r}^{(2)}_p(\lambda) = 0 < \epsilon$ for all eigenvalues $\lambda \in [a,b]$.

Next, $\hat{r}^{(i)}_p$ in $[c,d]$ is bounded by its maximum value within $[a,b]$ multiplied by the polynomial that is the fastest growing polynomial in $\mathcal{P}_{p}$ outside- and bounded below 1 within $[a,b]$. This polynomial is again the (transformed) Chebyshev polynomial $C_{p}\left(\frac{2x - b - a}{b - a}\right)$. Therefore,
\begin{equation*}
    \max_{\lambda \in [c,d]} |\hat{r}^{(i)}_p(\lambda)| \leq \eta_i C_{p}\left(\frac{2d - b - a}{b + a}\right),
\end{equation*}
with $\eta_i$ as defined in \cref{eq:residual_polynomial_bound_ab_i}.

At this point we have ensured that $\max_{\lambda \in [a,b]}|r_{\bar{m}}|$ is bounded by $\epsilon$ using \cref{eq:residual_polynomial_bound_ab}. So it remains to bound $\max_{\lambda \in [c,d]}|r_{\bar{m}}|$ in \cref{eq:residual_polynomial_bound_cd}. Using above results we can write 
\begin{equation*}
    \max_{\lambda \in [c,d]} |r_{\bar{m}}(\lambda)| < \epsilon,
\end{equation*}
if we require that
\begin{equation}
    \eta_i C_{p}\left(\frac{2d - b - a}{b - a}\right) /C_{\bar{m}-p}\left(\frac{d+c}{d-c}\right) < \epsilon.
    \label{eq:relative_error_bound_mp}
\end{equation}
Using that for $x_1, x_2, x_3 \in \mathbb{R}^+$ with $x_1 > x_3$ and $z = \frac{x_1 - x_2}{x_3}$
\begin{align*}
    C_p(z) & \leq \left(z + \sqrt{z^2 - 1}\right)^p \\
    & = \left( \frac{x_1 - x_2}{x_3} + \sqrt{ \left[\frac{x_1 - x_2}{x_3}\right]^2 -1}\right)^p \\
    & \leq \left( \frac{x_1}{x_3} + \sqrt{ \left[\frac{x_1}{x_3}\right]^2 - 1}\right)^p \\
    & \leq \left( \frac{2x_1}{x_3}\right)^p,
\end{align*}
and substituting $x_1 = 2d$, $x_2 = b + a$ and $x_3 = b - a$ we obtain the following inequality
\begin{equation}
    \eta_i \left(\frac{4d}{b-a} \right)^p /C_{\bar{m}-p}\left(\frac{d+c}{d-c}\right) < \epsilon. 
    \label{eq:chebyshev_degree_p_bound}
\end{equation}
Moreover,
\begin{align*}
    \eta_i \left(\frac{4d}{b-a}\right)^p &= 
    \begin{cases}
        2\left(\frac{\sqrt{b} - \sqrt{a}}{\sqrt{b} + \sqrt{a}} \frac{4d}{b-a}\right)^p &, \text{if } i = 1\\
        \left(\frac{b - a}{a}\frac{4d}{b-a}\right)^p &, \text{if } i = 2,
    \end{cases}\\
    &=
    \begin{cases}
        2\left(\frac{4d}{b + 2\sqrt{ab} + a}\right)^p &, \text{if } i = 1\\
        \left(\frac{4d}{a}\right)^p &, \text{if } i = 2,
    \end{cases}\\
    &\leq 2
    \begin{cases}
        \left(\frac{4d}{b}\right)^p &, \text{if } i = 1\\
        \left(\frac{4d}{a}\right)^p &, \text{if } i = 2,
    \end{cases}
\end{align*}
We can therefore require that the bound in \cref{eq:chebyshev_degree_p_bound} is satisfied if we have
\[
    1/C_{\bar{m}-p}\left(\frac{d+c}{d-c}\right) \leq \frac{\epsilon}{2\left( \frac{4d}{e_i}\right)^p},
\]
where 
\[
    e_i = \begin{cases}
        b &, \text{if } i = 1\\
        a &, \text{if } i = 2.\\
    \end{cases}
\]
Again using \cref{eq:chebyshev_polynomial_bound} and solving for the degree $\bar{m} - p$ we obtain
\[
    \bar{m} - p \geq \frac{1}{2}\sqrt{\frac{d}{c}}\left(\ln{\epsilon} + p \ln{\frac{4d}{e_i}}\right),
\]
which leads to the following bound for the number of iterations \cite[Equation 4.4]{cg_sharpened_convrate_Axelsson1976}
\begin{equation}
    \bar{m}=\left\lfloor\frac{1}{2} \sqrt{\frac{d}{c}} \ln (2 / \epsilon)+\left(1+\frac{1}{2} \sqrt{\frac{d}{c}} \ln (4 d / e_i)\right) p\right\rfloor,
    \label{eq:cg_iteration_bound_2_clusters}
\end{equation}
where 
\[
    1 \leq p \leq \min\left(\left\lfloor\frac{1}{2}\sqrt{\frac{b}{a}}\ln{\frac{2}{\epsilon}} + 1 \right\rfloor, N_{\text{tail}}\right).
\]

\section{Theoretical performance limit}\label{sec:cg_sharpened_convrate_th_performance}
In this section we obtain a theoretical lower limit to the improvement of the bound in \cref{eq:cg_iteration_bound_2_clusters} over the classical bound in \cref{eqeq:cg_convergence_rate_bound_iterations,eq:cg_convergence_rate_bound_iterations_approx}. Improvement is measured here as the ratio of the number of iterations predicted by the classical bound to that predicted by the sharpened bound
\begin{equation}
    P = \frac{m}{\bar{m}}.
    \label{eq:performance_ratio}
\end{equation}
In particular, we seek to find the minimum value of $P$. We know that the product of two lower order Chebyshev polynomials is not optimal for a uniform eigenspectrum, cf. the proof of Chebyshev optimality outlined in \cref{th:minmax_polynomial}. Therefore, a good place to start looking for a global minimum of $P$ is to study the case of a uniform eigenspectrum. This also reduces the parameter space in which we need to look for said minimum. Indeed, introducing the left and right cluster condition numbers $\kappa_l = \frac{b}{a}$ and $\kappa_r = \frac{d}{b}$, we can rewrite the bound in \cref{eq:cg_iteration_bound_2_clusters} as follows
\[
        \bar{m}(\kappa, \kappa_l, \kappa_r)=\left\lceil\frac{\sqrt{\kappa_r}}{2} \ln (2 / \epsilon)+\left(1+\frac{\sqrt{\kappa_r}}{2} \ln \left(\frac{4\kappa}{\kappa_l}\right)\right) p\right\rceil.
\]
Then, substituting $p$ gives
\begin{equation}
    \bar{m}=\left\lfloor
        1 
        + \frac{\sqrt{\kappa_r}}{2}\ln\left(\frac{4\kappa}{\kappa_l}\right)
        + \frac{1}{2}\ln\left(\frac{2}{\epsilon}\right)\left(
            \sqrt{\kappa_l}
            + \sqrt{\kappa_r}
            + \sqrt{\kappa_l\kappa_r}\ln\left(\frac{4\kappa}{\kappa_l}\right)
        \right)
    \right\rfloor.
\end{equation}  
Restricting ourselves to the case of a uniform eigenspectrum, i.e. $a<b=c<d$, we can set $\kappa=\kappa_l\kappa_r$, which yields
\begin{equation}
    \bar{m}=\left\lfloor
        1 
        + \frac{\sqrt{\kappa_r}}{2}\ln\left(4\kappa_r\right)
        + \underbrace{\frac{\sqrt{\kappa}}{2}\ln\left(\frac{2}{\epsilon}\right)}_{=m}\left(
            \frac{1}{\sqrt{\kappa_l}}
            + \frac{1}{\sqrt{\kappa_r}}
            + \ln\left(4\kappa_r\right)
        \right)
    \right\rfloor.
    \label{eq:cg_iteration_bound_2_clusters_uniform}
\end{equation}
We recognize the classical CG bound in the last factor of \cref{eq:cg_iteration_bound_2_clusters_uniform}. Now, the performance ratio $P$ in \cref{eq:performance_ratio} can be approximated as follows
\begin{equation}
    P(\kappa_l, \kappa_r) = \left(\frac{1}{\sqrt{\kappa_l}} + \frac{1}{\sqrt{\kappa_r}} + \ln(4\kappa_r)\right)^{-1} = 
\end{equation}
- Note, $m_c \sim \sqrt{\kappa}\ln(2/\epsilon)$. Hence, the minimal performance satisfies
\[
p_{\text{min}}(\kappa_l, \kappa_r) = \min_{\kappa, \ \kappa_r=c, \kappa_l = \frac{\kappa}{\kappa_r}} m_c/m_s = \left(\frac{1}{2\sqrt{\kappa_r}} + \frac{1}{4}\ln(4\kappa_r) + \sqrt{\kappa_l}\ln\left(\frac{\kappa_r}{4}\right)\right)^{-1},
\]
Now, setting $q(\kappa_l,\kappa_r) = \frac{1}{4}\ln(4\kappa_r) + \sqrt{\kappa_l}\ln\left(\frac{\kappa_r}{4}\right)$ and using the Taylor expansion for $\frac{1}{1 + x} \approx 1 - x + \mathcal{O}(x^2)$ for small $x = \frac{1}{2q(\kappa_l,\kappa_r)\sqrt{\kappa_r}}$ we get
\[
p_{\text{min}}(\kappa_l, \kappa_r) = q(\kappa_l,\kappa_r)^{-1} - \frac{1}{2q(\kappa_l,\kappa_r)^2\sqrt{\kappa_r}} + \mathcal{O}\left(\frac{1}{4q(\kappa_l,\kappa_r)^3\kappa_r}\right)
\]
- Now expanding $q^{-1}(\kappa_l,\kappa_r)$ with the same Taylor expansion and $x=\frac{\ln(4\kappa_r)}{4\sqrt{\kappa_l}\ln\frac{\kappa_r}{4}}$
\[
q(\kappa_l,\kappa_r)^{-1} = (\sqrt{\kappa_l}\ln(4\kappa_r))^{-1} - \frac{\ln(4\kappa_r)}{4\kappa_l\ln(\frac{\kappa_r}{4})^2} + \mathcal{O}\left(\frac{\ln(4\kappa_r)}{16\kappa^{3/2}_1\ln(\frac{\kappa_r}{4})^3}\right)
\]
- Combining the two expansions we get
\[
p_{\text{min}}(\kappa_l, \kappa_r) = (\sqrt{\kappa_l}\ln(4\kappa_r))^{-1} + \mathcal{O}\left(\frac{1}{2\kappa_l\sqrt{\kappa_r}\ln(4\kappa_r)^2}\right)
\]
- Since $\kappa_l,\kappa_r \geq 1$ we neglect the higher order terms  we get
\[
p_{\text{min}}(\kappa_l, \kappa_r) \approx (\sqrt{\kappa_l}\ln(4\kappa_r))^{-1} \tag{2}
\]
- Eq. (2) explains the performance curve for the uniform spectra in the performance plot, as the minimum performance is largely independent of condition number. Only for small $\kappa_l \sim \kappa$ do we notice a slight increase (notice the little increase of the same performance curve for small $\kappa$). Additionally, Eq. (2) shows that for increasing $\kappa_r$ we expect a decreasing minimum performance $p_{\text{min}}$.

\section{Generalization to multiple clusters}\label{sec:multiple_clusters}
At this point we assume that we are dealing with an eigenspectrum of the form $\sigma_1(A)$, i.e. we are only treating case 1. In \cref{sec:cg_sharpened_convrate_implications}, it is reasoned that this is indeed a very applicable case for \cref{prob:elliptic_problem_discretized}.

In this case, the technique outlined in \cref{sec:cg_sharpened_convrate} starts at the left most cluster $[a,b]$, finds the Chebyshev degree $p_1=p$ satisfying \cref{eq:chebyshev_degree_p}, moves to the neighboring cluster $[c,d]$ and finds the Chebyshev degree $p_2 = \bar{m} - p$ satisfying \cref{eq:relative_error_bound_mp}. Rewriting \cref{eq:relative_error_bound_mp} gives the following equation for $p_2$:
\begin{equation}
    \frac{1}{C_{p_2}\left(\frac{d+c}{d-c}\right)} \leq \frac{\epsilon}{{C}^{(1)}_{p_1}(d)} = \epsilon_2,
    \label{eq:chebyshev_degree_p_prime}
\end{equation}
where
\[
    C^{(1)}_{p_1}(x) = C_{p_1}\left(\frac{b + a - 2x}{b - a}\right) /C_{p_1}\left(\frac{b+a}{b-a}\right),
\]
is the Chebyshev polynomial corresponding to the first cluster.

Suppose there is a third cluster to the right of $[c,d]$, i.e. $[e,f]$. We can repeat the process and find the Chebyshev degree $p_3$ satisfying a similar as \cref{eq:chebyshev_degree_p_prime} for the third cluster. 
\[
    \frac{1}{C_{p_3}\left(\frac{f+e}{f-e}\right)} \leq \frac{\epsilon}{C^{(1)}_{p_1}(f)C^{(2)}_{p_2}(f)} = \epsilon_3,
\]
This leads to the general equation for the Chebyshev degree $p_i$ of the $i^{\text{th}}$ cluster $[a_i, b_i]$
\begin{equation}
    \frac{1}{C_{p_i}\left(\frac{b_i + a_i}{b_i - a_i}\right)} \leq \frac{\epsilon}{\prod_{j=1}^{i-1} C^{(j)}_{p_j}(b_i)} = \epsilon_i.
    \label{eq:chebyshev_degree_p_i}
\end{equation}

Due to the large range of the Chebyshev polynomials $\tilde{C}_p$ computer is likely to result in floating point number overflow during calculation of the denominator of \cref{eq:chebyshev_degree_p_i}. Instead, we first apply \cref{eq:chebyshev_polynomial_bound} and introduce the cluster condition numbers $\kappa_i = \frac{b_i}{a_i}$, where $i$ is the index of the cluster. We can then rewrite \cref{eq:chebyshev_degree_p_i} as follows
\begin{equation*}
    p_i  =  \left\lceil\ln{\frac{\epsilon_i}{2}} / \ln{\frac{\sqrt{\kappa_i} - 1}{\sqrt{\kappa_i} + 1}}\right\rceil, \\
\end{equation*}
and
\begin{align*}
    \ln{\frac{\epsilon_i}{2}} & = \ln{\frac{\epsilon}{2}} - \sum_{j=1}^{i-1} \ln{C^{(j)}_{p_j}(b_i)}. \\
\end{align*}
Let $z^{(i,j)}_1 = \frac{b_j + a_j - 2b_i}{b_j - a_j}$ and $z^{(j)}_2 = \frac{b_j + a_j}{b_j - a_j}$ then
\begin{equation*}
    \ln{C^{(j)}_{p_j}(b_i)} = \ln{C_{p_j}(z^{(i,j)}_1)} - \ln{C_{p_j}(z^{(j)}_2)}.
\end{equation*}
We have, using the definition of the Chebyshev polynomial
\begin{equation}
    \ln{C_{p_j}(z^{(i,j)}_1)} \lessapprox p_j \ln{\left[z^{(i,j)}_1 - \sqrt{\left(z^{(i,j)}_1\right)^2 - 1}\right]} - \ln{2},
    \label{eq:chebyshev_polynomial_bound_z1}
\end{equation}
and
\begin{equation}
    \ln{C_{p_j}(z^{(j)}_2)} \gtrapprox p_j \ln{\left[z^{(j)}_2 + \sqrt{\left(z^{(j)}_2\right)^2 - 1}\right]} - \ln{2},
    \label{eq:chebyshev_polynomial_bound_z2}
\end{equation}
both of which become more accurate approximations as $z,m\rightarrow\infty$. Introducing 
\begin{align*}
    \zeta^{(i,j)}_1 &= z^{(i,j)}_1 - \sqrt{\left(z^{(i,j)}_1\right)^2 - 1}, \\
    \zeta^{(j)}_2 &= z^{(j)}_2 + \sqrt{\left(z^{(j)}_2\right)^2 - 1}, \text{ and}\\
    f_i &= \frac{\sqrt{\kappa_i} - 1}{\sqrt{\kappa_i} + 1},
\end{align*}
with $\kappa_i$ the $i^{\text{th}}$ cluster condition number, and substituting the inequalities \ref{eq:chebyshev_polynomial_bound_z1} and \ref{eq:chebyshev_polynomial_bound_z2} back into the bound for $p_i$ gives
\begin{align*}
    p_i &\leq \left\lceil\frac{\ln{\frac{\epsilon}{2}} - \sum_{j=1}^{i-1} p_j\left(\ln{\zeta^{(i,j)}_1} - \ln{\zeta^{(j)}_2} \right)}{\ln{f_i}}\right\rceil \\
    &= \left\lceil\log_{f_i}{\frac{\epsilon}{2}} - \sum_{j=1}^{i-1} p_j\left(\log_{f_i}{\zeta^{(i,j)}_1} - \log_{f_i}{\zeta^{(j)}_2} \right)\right\rceil\\
    &= \left\lceil\log_{f_i}{\frac{\epsilon}{2}} - \sum_{j=1}^{i-1} p_j\log_{f_i}{\frac{\zeta^{(i,j)}_1}{\zeta^{(j)}_2}} \right\rceil
\end{align*}
Note that in general $\zeta^{(i,j)}_1 < \zeta^{(j)}_2$ and hence $\log_{f_i}{\left(\frac{\zeta^{(j)}_2}{\zeta^{(i,j)}_1}\right)} > 0$. This prompts us to write
\begin{equation}
    p_i \leq \left\lceil\log_{f_i}{\frac{\epsilon}{2}} + \sum_{j=1}^{i-1} p_j\log_{f_i}{\frac{\zeta^{(j)}_2}{\zeta^{(i,j)}_1}} \right\rceil
    \label{eq:chebyshev_degree_p_i_explicit}
\end{equation}
Evidently, adding more clusters to the left of the interval $[a_i,b_i]$ increases the degree $p_i$ of the Chebyshev polynomial. Next to this, \cref{eq:chebyshev_degree_p_i_explicit} reduces to the classical CG iteration bound \cref{eq:cg_convergence_rate_bound_iterations} for a single cluster when $i = N_{\text{clusters}} = 1$.

Equation \ref{eq:chebyshev_degree_p_i_explicit} gives us a way to calculate the Chebyshev degree $p_i$ of the $i^{\text{th}}$ cluster $[a_i,b_i]$ in terms of the Chebyshev degrees of the previous clusters. To obtain a bound on the number of iterations for the CG method we sum the Chebyshev degrees of all the clusters
\begin{equation}
    \bar{m} = \sum_{i=1}^{N_{\text{clusters}}} p_i
    \label{eq:cg_iteration_bound_multiple_clusters}
\end{equation}