\chapter{Conclusion}\label{ch:conclusion}
This thesis has stressed that the classical condition number-based Conjugate Gradient (CG) iteration bound, $m_1$, from \cref{eq:cg_convergence_rate_bound_iterations_approx}, does not fully capture the convergence behavior in high-contrast heterogeneous elliptic problems, particularly when two-level Schwarz preconditioners are employed. The spectra of the preconditioned systems, $\sigma(M^{-1}A)$ for all $M^{-1}\in\mathcal{M}^{-1}$ as defined in \cref{eq:preconditioners}, were found in \cite{ams_coarse_space_comp_study_Alves2024} to not only possess a condition number like that of \cref{eq:msfem_condition_number} but also to exhibit the spectral gap discussed in \cref{sec:spectral_gap_darcy}. The presence of this spectral gap undermines the assumption of a uniformly distributed eigenspectrum, a key premise in the derivation of the classical CG iteration bound. This observation forms the primary motivation for this thesis: the classical bound is too coarse for high-contrast problems, necessitating the development of sharper, more descriptive bounds.

\section{Development of Sharpened Iteration Bounds}
A review of the relevant literature in \cref{sec:cg_nonuniform_spectra} revealed that a two-cluster sharpened CG iteration bound was derived in \cite[Section 4]{cg_sharpened_convrate_Axelsson1976}, expressed in terms of the four extremal eigenvalues of the two clusters within a spectrum. In \cref{sec:cg_sharpened_convrate}, this bound was re-derived. The necessary conditions for which this two-cluster bound, $m_2$, is sharper than the classical bound, $m_1$, were established in \cref{sec:performance_ratio,sec:tail_cluster_bound_performance} through \cref{eq:threshold_inequality_explicit_expansion,eq:tail_cluster_bound_sparsity_condition}. This process identified the three key spectral characteristics sought by \ref{rq:subsidiary:measures}: the left- and right-cluster condition numbers, $\kappa_l$ and $\kappa_r$, and the spectral width, $s = \frac{\kappa}{\kappa_l}$.

Building on the work of \citeauthor{cg_sharpened_convrate_Axelsson1976}, we developed two novel sharpened CG iteration bounds in \cref{sec:cg_iteration_bound_algorithm}. By combining the aforementioned necessary conditions based on $\kappa_l,\kappa_r$ and $s$ ensuring $m_2 < m_1$ with a recursive partitioning algorithm and a generalized multi-cluster version of the bound from \cite{cg_sharpened_convrate_Axelsson1976}, we introduced a multi-cluster bound, $m_{N_{\text{cluster}}}$ (\cref{alg:multi_cluster_cg_bound}), and a tail-cluster bound, $m_{N_{\text{tail-cluster}}}$ (\cref{alg:multi_tail_cluster_cg_bound}). The only difference between these two bounds is that the tail-cluster bound treats smaller, sparse clusters in a spectrum differently by considering each eigenvalue in that cluster individually, rather than as part of a collective cluster. This distinction allows the tail-cluster bound to capture more detailed spectral information.

\section{Numerical Validation and Performance}
The sharpness of these new bounds was investigated in \cref{sec:sharpness_of_bounds} by applying them to approximate eigenspectra obtained from the Lanczos matrix. These spectra resulted from a converged PCG process with relative residual error stopping-criterion of $\epsilon_r=10^{-8}$ applied to \cref{prob:elliptic_problem_discretized} for three two-level Schwarz preconditioners constructed with GDSW, RGDSW, or AMS coarse spaces. The bounds were evaluated for two specific high-contrast scalar coefficient functions, $\mathcal{C}_{\text{3layer, vert}}$ and $\mathcal{C}_{\text{edge slabs, around vertices}}$, as shown in \cref{fig:coefficient_functions}.

In all tested scenarios, the tail-cluster bound, $m_{N_{\text{tail-cluster}}}$, consistently outperformed the multi-cluster bound, $m_{N_{\text{cluster}}}$, which in turn was significantly sharper than the classical condition number-based bound, $m_1$. Specifically, we found that both new bounds are either of the same order as $m$, that is $m_{N_{\text{cluster}}}, m_{N_{\text{tail-cluster}}} = \mathcal{O}(m)$, or one order higher, $m_{N_{\text{cluster}}}, m_{N_{\text{tail-cluster}}} = \mathcal{O}(10m)$. This result positively answers \ref{rq:subsidiary:heuristic} and demonstrates the superior accuracy of the newly developed bounds. Furthermore, both $m_{N_{\text{cluster}}}$ and $m_{N_{\text{tail-cluster}}}$ provide valuable information about the convergence behavior and can distinguish between the robustness of different preconditioners more accurately than the classical bound.

\section{Challenges in Practical Estimation and Future Directions}
Despite their sharpness, we showed in \cref{sec:early_estimation_of_iterations} that the practical application of $m_{N_{\text{cluster}}}$ and $m_{N_{\text{tail-cluster}}}$ for \textit{a priori} iteration estimation faces challenges. The bounds require more detailed spectral information than is typically available or computationally feasible to obtain from the initial PCG iterations. The core issue is that the Ritz values may not converge quickly enough to the true eigenvalues of $A$, particularly the internal ones defining cluster boundaries, to provide an accurate estimate of the full spectrum in the early PCG iterations. Consequently, the answer to \ref{rq:subsidiary:preconditioners} is that the utility of these bounds for early estimation depends on the specific coefficient function and preconditioner used.

In fact, the classical bound $m_1$ suffers from slowly converging Ritz values as well. Though to a lesser extent than the new bounds. This is because $m_1$ only relies on the extremal eigenvalues of the spectrum. In contrast, $m_{N_{\text{cluster}}}$ is most accurate when the extremal eigenvalues of \textit{each cluster} are well approximated by the Ritz values, which is not always the case. Similarly, $m_{\text{tail-cluster}}$ requires the extremal eigenvalues of each cluster as well as a set of specific tail eigenvalues to be well approximated by the Ritz values. This is a more stringent requirement, which explains why $m_{N_{\text{tail-cluster}}}$ is not always an upper bound for $m$ when only a few Ritz values are available.

For most combinations of coefficient functions and meshes tested with the GDSW and AMS coarse spaces, we observed the following relationship within the first 300 PCG iterations:
\[
    m_{\text{tail-cluster}}(\sigma(T_i)) \lesssim m \lesssim m_{N_{\text{cluster}}}(\sigma(T_i)) \text{ for } i \leq N_{\text{iter}} = 300,
\]
where $m$ is the actual number of iterations and $\sigma(T_i)$ is the Ritz spectrum at iteration $i$. This suggests we can leverage the tail-cluster bound to obtain a more accurate estimate of the number of iterations required for convergence. The average of the multi-cluster and tail-cluster bounds at iteration $i$ denoted as $m_{\text{estimate}}$ is such a heuristic. Though $m_{\text{estimate}}$ is arbitrarily constructed and there is no guarantee that it is a valid upper bound, it provides good estimates of $m$. However, even the performance of $m_{N_{\text{cluster}}}$ as an upper bound and $m_{\text{estimate}}$ as a heuristic deteriorates when the RGDSW coarse space is used to solve \cref{prob:elliptic_problem_discretized} with $\mathcal{C}_{\text{edge slabs, around vertices}}$, where they can underestimate the true iteration count.

In conclusion, the main goal of this thesis was to sharpen the CG iteration bound for Schwarz-preconditioned high-contrast heterogeneous scalar-elliptic problems beyond the classical condition number-based bound. The derived multi-cluster and tail-cluster bounds offer a more nuanced and accurate picture of convergence behavior than the classical condition number-based bound, able to distinguish between preconditioners effectively.

Future research should focus on two main areas. First, applying the new bounds to a wider range of problems, including those with more complex high-contrast coefficients, finer mesh discretizations, different preconditioners, and other types of PDEs. Second, and more fundamentally, research into the \textit{a priori}  estimation of the key spectral characteristics ($\kappa_l, \kappa_r, s$) is crucial to circumvent the dependency of the new bounds on the slowly converging Ritz values, which would unlock their full potential for predictive performance analysis.


% This thesis stresses that the classical condition number-based CG iteration bound $m_1$ from \cref{eq:cg_convergence_rate_bound_iterations_approx} does not fully capture the convergence behavior in high-contrast heterogeneous elliptic problems, particularly when two-level Schwarz preconditioners are employed. The spectra of the preconditioned systems $\sigma(M^{-1}A), \ \forall M^{-1}\in\mathcal{M}^{-1}$ with $\mathcal{M}^{-1}$ as in \cref{eq:preconditioners}, were found in \cite{ams_coarse_space_comp_study_Alves2024} to not only have a condition number like that of \cref{eq:msfem_condition_number}, but also exhibit the spectral gap discussed in \cref{sec:spectral_gap_darcy}. The presence of this spectral gap undermines the assumption that the eigenspectrum of the preconditioned system is uniformly distributed, which is a key assumption in deriving the classical CG iteration bound. This leads to the main motivation of this thesis that the classical bound is too rough for high-contrast problems.

% A study of the relevant literature in \cref{sec:cg_nonuniform_spectra} reveals that a two-cluster sharpened CG iteration bound was derived in \cite[Section 4]{cg_sharpened_convrate_Axelsson1976} in terms of the four extremal eigenvalues of the two clusters in a given spectrum. In \cref{sec:cg_sharpened_convrate}, this bound is derived once again. The necessary conditions, \cref{eq:threshold_inequality_explicit_expansion,eq:tail_cluster_bound_sparsity_condition}, for which $m_2 < m_1$ are derived in \cref{sec:performance_ratio,sec:tail_cluster_bound_performance} and in the process we obtain the three spectral characteristics that \cref{rq:subsidiary:measures} seeks to identity: the left- and right cluster condition numbers $\kappa_l, \kappa_r$ as well as the spectral width $s = \frac{\kappa}{\kappa_l}$.

% Using the aforementioned necessary conditions based on $\kappa_l,\kappa_r$ and $s$ ensuring $m_2 < m_1$ in combination with a recursive partitioning algorithm and a generalized multi-cluster version of the bound from \cite{cg_sharpened_convrate_Axelsson1976}, we obtain in \cref{sec:cg_iteration_bound_algorithm} two new sharpened CG iteration bounds: a multi-cluster bound \cref{alg:multi_cluster_cg_bound}, $m_{N_{\text{cluster}}}$, and a tail-cluster bound \cref{alg:multi_tail_cluster_cg_bound}, $m_{N_{\text{tail-cluster}}}$. 

% The sharpness of these bounds is investigated in \cref{sec:sharpness_of_bounds} by applying them to approximate eigenspectra obtained from the Lanczos matrix, resulting from a converged PCG method applied to \cref{prob:elliptic_problem_discretized} for three two-level Schwarz preconditioners constructed with either a GDSW, RGDSW or AMS coarse space. The bounds are calculated for two specific high-contrast scalar coefficient functions $\mathcal{C}$ given in \cref{fig:coefficient_functions}. In all cases the tail-cluster bound $m_{N_{\text{tail-cluster}}}$ outperforms the multi-cluster bound $m_{N_{\text{cluster}}}$, which in turn outperforms the classical condition number-based bound $m_1$, answering \ref{rq:subsidiary:heuristic} positively.

% In \cref{sec:early_estimation_of_iterations} we show that though the new bounds $m_{N_{\text{cluster}}}$ and $m_{N_{\text{tail-cluster}}}$ are sharper than the classical bound $m_1$, they may need more information about the spectrum than is available \textit{a priori}  or computationally feasible to obtain by doing the first few PCG iterations. The culprit being in this case that the rate of convergence of the Ritz values to the true eigenvalues may not be fast enough to obtain a good estimate of the full spectrum of $A$. Hence, the answer to \ref{rq:subsidiary:preconditioners} is that it depends on the specific coefficient function and preconditioner used. 

% In particular, for almost all combinations of coefficient functions and meshes in this thesis the multi-cluster bound $m_{N_{\text{cluster}}}$ gives an upper bound that is at most 4 times larger than the actual number of iterations required for convergence $m$, which is based on Ritz values obtained within the first 300 PCG iterations and only for the Schwarz preconditioners with the GDSW and AMS coarse spaces. For these cases, we make the following observation
% \[
%     m_{\text{tail-cluster}}(\sigma(T_i)) \lesssim m \lesssim m_{N_{\text{cluster}}}(\sigma(T_i)) \text{ for } i \leq N_{\text{iter}} = 300,
% \]
% where $\sigma(T_i)$ denotes the spectrum or Ritz values of the preconditioned system at iteration $i$. This observation suggests we can leverage the tail-cluster bound to obtain a more accurate estimate of the number of iterations required for convergence. The average of the multi-cluster and tail-cluster bounds at iteration $i$ denoted as $m_{\text{estimate}}$ is such a heuristic. Though $m_{\text{estimate}}$ is arbitrarily constructed and there is no guarantee that it is a valid upper bound, it provides a practical way to gauge the convergence behavior.

% However, even $m_{N_{\text{cluster}}}$ deteriorates as an upper bound when the RGDSW coarse space is applied to the \cref{prob:elliptic_problem_discretized} with $\mathcal{C}_{\text{edge slabs, around vertices}}$. In this case, $m_{N_{\text{cluster}}}$ underestimates $m$, where the difference grows with increasing number of subdomains.

% That being said, both bounds $m_{N_{\text{cluster}}}$ and $m_{N_{\text{tail-cluster}}}$ provide valuable information about the convergence behavior of the (P)CG method and can distinguish between preconditioners more effectively than the classical bound. In fact, the classical bound $m_1$ suffers from slowly converging Ritz values as well. Though to a lesser extent than the new bounds. This is because $m_1$ only relies on the extremal eigenvalues of the spectrum. In contrast, $m_{N_{\text{cluster}}}$ is most accurate when the extremal eigenvalues of each cluster are well approximated by the Ritz values, which is not always the case. Similarly, $m_{\text{tail-cluster}}$ requires the extremal eigenvalues of each cluster as well as a set of specific tail eigenvalues to be well approximated by the Ritz values. This is a more stringent requirement, which explains why $m_{N_{\text{tail-cluster}}}$ is not always an upper bound for $m$ when only a few Ritz values are available. 

% In conclusion, the main goal of this thesis was to sharpen the CG iteration bound for Schwarz-preconditioned high-contrast heterogeneous scalar-elliptic problems beyond the classical condition number-based bound. This was achieved by deriving a two-cluster sharpened CG iteration bound and a tail-cluster sharpened CG iteration bound, which outperform the classical bound in terms of sharpness. The results show that the spectral characteristics of the preconditioned system play a crucial role in determining the convergence behavior of the (P)CG method, and that these characteristics can be used to obtain sharper bounds than the classical condition number-based bound. Further research directions can include applying the new bounds $m_{N_{\text{cluster}}}$ and $m_{N_{\text{tail-cluster}}}$ to spectra obtained from scalar elliptic problems with more complex high-contrast coefficient functions, discretizations with finer meshes and different types of preconditioners and other kinds of PDEs. Finally, more fundamental research on the \textit{a priori} estimation of spectral characteristics like $\kappa_l,\kappa_r$ and $s$ is necessary to circumvent the dependence of $m_{N_{\text{cluster}}}$ and $m_{N_{\text{tail-cluster}}}$ on slowly convergent Ritz values.
