\chapter{Conclusion}\label{ch:conclusion}
This thesis has stressed that the classical condition number-based Conjugate Gradient (CG) iteration bound, $m_1$, from \cref{eq:cg_convergence_rate_bound_iterations_approx}, does not fully capture the convergence behavior in high-contrast heterogeneous elliptic problems, particularly when two-level Schwarz preconditioners are employed. The spectra of the preconditioned systems, $\sigma(M^{-1}A)$ for all $M^{-1}\in\mathcal{M}^{-1}$ as defined in \cref{eq:preconditioners}, were found in \cite{ams_coarse_space_comp_study_Alves2024} to not only possess a condition number like that of \cref{eq:msfem_condition_number} but also to exhibit the spectral gap discussed in \cref{sec:spectral_gap_darcy}. The presence of this spectral gap undermines the assumption of a uniformly distributed eigenspectrum, a key premise in the derivation of the classical CG iteration bound. This observation forms the primary motivation for this thesis: the classical bound is too coarse for high-contrast problems, necessitating the development of sharper, more descriptive bounds.

The main contributions of this thesis are the development and analysis of sharpened Conjugate Gradient (CG) iteration bounds for high-contrast heterogeneous elliptic problems preconditioned by two-level Schwarz methods. Building on the ideas from \cite[Section 4]{cg_sharpened_convrate_Axelsson1976}, this work introduces new multi-cluster and tail-cluster bounds that more accurately reflect the convergence behavior in the presence of clustered eigenspectra. These bounds are derived, generalized, and validated both theoretically and numerically, and their performance is compared to the classical condition number-based bound. The thesis also presents practical algorithms for partitioning eigenspectra and computing these sharpened bounds, and discusses the challenges and future directions for their application in predictive performance analysis.

\section{Development of Sharpened Iteration Bounds}
A review of the relevant literature in \cref{sec:cg_nonuniform_spectra} revealed that a two-cluster sharpened CG iteration bound was derived in \cite[Section 4]{cg_sharpened_convrate_Axelsson1976}, expressed in terms of the four extremal eigenvalues of the two clusters within a spectrum. In \cref{sec:cg_sharpened_convrate}, this bound was re-derived. The necessary conditions for which this two-cluster bound, $m_2$, is sharper than the classical bound, $m_1$, were established in \cref{sec:performance_ratio,sec:tail_cluster_bound_performance} through \cref{eq:threshold_inequality_explicit_expansion,eq:tail_cluster_bound_sparsity_condition}. This process identified the three key spectral characteristics sought by \ref{rq:subsidiary:measures}: the left- and right-cluster condition numbers, $\kappa_l$ and $\kappa_r$, and the spectral width, $s = \frac{\kappa}{\kappa_l}$.

We developed two novel sharpened CG iteration bounds in \cref{sec:cg_iteration_bound_algorithm}. By combining the aforementioned necessary conditions based on $\kappa_l,\kappa_r$ and $s$ ensuring $m_2 < m_1$ with a recursive partitioning algorithm and a generalized multi-cluster version of the bound from \cite{cg_sharpened_convrate_Axelsson1976}, we introduced a multi-cluster bound, $m_{N_{\text{cluster}}}$ (\cref{alg:multi_cluster_cg_bound}), and a tail-cluster bound, $m_{N_{\text{tail-cluster}}}$ (\cref{alg:multi_tail_cluster_cg_bound}). The only difference between these two bounds is that the tail-cluster bound treats smaller, sparse clusters in a spectrum differently by considering each eigenvalue in that cluster individually, rather than as part of a collective cluster. This distinction allows the tail-cluster bound to capture more detailed spectral information.

\section{Numerical Validation and Performance}
The sharpness of these new bounds was investigated in \cref{sec:sharpness_of_bounds} by applying them to approximate eigenspectra obtained from the tridiagonal Lanczos matrix. These spectra resulted from a converged PCG process with relative residual error stopping-criterion of $\epsilon_r=10^{-8}$ applied to \cref{prob:elliptic_problem_discretized} for three two-level Schwarz preconditioners constructed with GDSW, RGDSW, or AMS coarse spaces. The bounds were evaluated for two specific high-contrast scalar coefficient functions, $\mathcal{C}_{\text{3layer, vert}}$ and $\mathcal{C}_{\text{edge slabs, around vertices}}$, as shown in \cref{fig:coefficient_functions}.

In all tested scenarios, the tail-cluster bound, $m_{N_{\text{tail-cluster}}}$, consistently outperformed the multi-cluster bound, $m_{N_{\text{cluster}}}$, which in turn was significantly sharper than the classical condition number-based bound, $m_1$. With regard to the relative improvement, we found that the new bounds were anywhere between 10 to 1000 times smaller than the classical bound. In an absolute sense, we found that both new bounds are either of the same order as $m$, that is $m_{N_{\text{cluster}}}, m_{N_{\text{tail-cluster}}} = \mathcal{O}(m)$, or one order higher, $m_{N_{\text{cluster}}}, m_{N_{\text{tail-cluster}}} = \mathcal{O}(10m)$. This result positively answers \ref{rq:subsidiary:heuristic} and demonstrates the superior accuracy of the newly developed bounds. Furthermore, both $m_{N_{\text{cluster}}}$ and $m_{N_{\text{tail-cluster}}}$ provide valuable information about the convergence behavior and can distinguish between the robustness of different preconditioners more accurately than the classical bound.

\section{Challenges in Practical Estimation and Future Directions}
Despite their sharpness, we showed in \cref{sec:early_estimation_of_iterations} that the practical application of $m_{N_{\text{cluster}}}$ and $m_{N_{\text{tail-cluster}}}$ for \textit{a priori} iteration estimation faces challenges. The bounds require more detailed spectral information than is typically available or computationally feasible to obtain from the initial PCG iterations. The core issue is that the Ritz values may not converge quickly enough to the true eigenvalues of $A$, particularly the internal ones defining cluster boundaries, to provide an accurate estimate of the full spectrum in the early PCG iterations. Consequently, the answer to \ref{rq:subsidiary:preconditioners} is that the utility of these bounds for early estimation depends on the specific coefficient function and preconditioner used.

In fact, the classical bound $m_1$ suffers from slowly converging Ritz values as well. Though to a lesser extent than the new bounds. This is because $m_1$ only relies on the extremal eigenvalues of the spectrum. In contrast, $m_{N_{\text{cluster}}}$ is most accurate when the extremal eigenvalues of \textit{each cluster} are well approximated by the Ritz values, which is not always the case. Similarly, $m_{\text{tail-cluster}}$ requires the extremal eigenvalues of each cluster as well as a set of specific tail eigenvalues to be well approximated by the Ritz values. This is a more stringent requirement, which explains why $m_{N_{\text{tail-cluster}}}$ is not always an upper bound for $m$ when only a few Ritz values are available.

For most combinations of coefficient functions and meshes tested with the GDSW and AMS coarse spaces, we observed the following within the first 300 PCG iterations:
\begin{equation}
    m_{\text{tail-cluster}}(\sigma(T_i)) \lesssim m \lesssim m_{N_{\text{cluster}}}(\sigma(T_i)) \text{ for } i \leq N_{\text{iter}} = 300,
    \label{eq:heuristic_observation}
\end{equation}
where $m$ is the actual number of iterations and $\sigma(T_i)$ is the Ritz spectrum at iteration $i$. Note that this relationship is not guaranteed to persist for higher values of $N_{\text{iter}}$ the RGDSW coarse space or other variants of \cref{prob:elliptic_problem_weak}.

That being said, \cref{eq:heuristic_observation} suggests we can leverage the tail-cluster bound to obtain a more accurate estimate of the number of iterations required for convergence. The average of the multi-cluster and tail-cluster bounds at iteration $i$ denoted as $m_{\text{estimate}}$ is such a heuristic. Though $m_{\text{estimate}}$ is arbitrarily constructed and there is no guarantee that it is a valid upper bound, it provides good estimates of $m$. However, even the performance of $m_{N_{\text{cluster}}}$ as an upper bound and $m_{\text{estimate}}$ as a heuristic deteriorates when the RGDSW coarse space is used to solve \cref{prob:elliptic_problem_discretized} with $\mathcal{C}_{\text{edge slabs, around vertices}}$, where they can underestimate the true iteration count.

In conclusion, the main goal of this thesis was to sharpen the CG iteration bound for Schwarz-preconditioned high-contrast heterogeneous scalar-elliptic problems beyond the classical condition number-based bound. The derived multi-cluster and tail-cluster bounds offer a more nuanced and accurate picture of convergence behavior than the classical condition number-based bound, able to distinguish between preconditioners effectively.

Future research could focus on three main areas. First, the splitting conditions derived in \cref{sec:performance_ratio,sec:tail_cluster_bound_performance} could be refined to better identify advantageous cluster partitions, potentially enhancing the sharpness of the multi-cluster and tail-cluster bounds as well as countering the effects of slowly converging Ritz values.  Second, and more fundamentally, research into the \textit{a priori} estimation of the key spectral characteristics ($\kappa_l, \kappa_r, s$) or all internal, extremal eigenvalues is crucial to circumvent the dependency of specifically $m_{N_{\text{cluster}}}$ on an approximate Ritz spectrum altogether, which would unlock its full potential for predictive performance analysis. Finally, it would be beneficial to apply the new bounds to a wider range of problems, including those with more complex, multi-high-contrast coefficients, finer mesh discretizations, different preconditioners, and other types of PDEs. This added complexity should come at no extra theoretical cost, as the bounds are derived for general SPD matrices and can handle multiple clusters. Such studies would further validate the robustness and versatility of the new bounds in practical applications.