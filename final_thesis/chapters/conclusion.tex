\chapter{Conclusion}\label{ch:conclusion}
This thesis stresses that the classical condition number-based CG iteration bound does not fully capture the convergence behavior in high-contrast heterogeneous elliptic problems, particularly when Schwarz preconditioners are employed. By incorporating additional spectral characteristics, such as the distribution, clustering, and gaps of eigenvalues in the eigenspectrum, the refined bound offers a more accurate and discriminative measure of iterative performance. Preliminary results indicate that these spectral properties significantly influence the convergence rate and can be exploited to predict and compare the efficacy of different preconditioners. 

Further research is required to test the refined bound on an eigenspectrum typically found in Darcy problems and to develop robust methods for estimating the full eigenspectrum in high-contrast problems found in practice. This research contributes to the theoretical understanding of iterative solvers and has practical importance for improving computational efficiency in solving complex elliptic PDEs.

\section{Implications for research}\label{sec:cg_sharpened_convrate_implications}
The preliminary results discussed in this chapter show that we can find both an a priori analytic two-cluster (\cref{eq:cg_iteration_bound_2_clusters}) and multiple-cluster (\cref{eq:cg_iteration_bound_multiple_clusters}) sharpened iteration bound for the CG method. The latter can also be implemented as a sequential, numerical algorithm that can be applied to artificially constructed spectra in \cref{fig:cg_sharpened_bound}. The sharpened bound appears to perform best in the two-cluster case, which has the most correspondence to the eigenspectrum of a typical (preconditioned) Darcy problem. Hence, \ref{rq:subsidiary:heuristic} is answered positively. 

With regard to \ref{rq:subsidiary:measures}, the cluster condition number $\kappa_i$ is introduced as a measure of the cluster width. This suggests that we can use the cluster condition number to distinguish between different preconditioners which is a promising result for \ref{rq:subsidiary:preconditioners}. Moreover, inspecting \cref{eq:cg_iteration_bound_2_clusters} more closely for the eigenspectrum in \cref{eq:two_clusters} (case $i=1$) reveals the presence of a sort of spectral gap $\frac{d}{b}$. This serves as yet another candidate for a potential set of spectral characteristics to estimate the distribution of eigenvalues in the eigenspectrum.

A logical next step is to more rigorously investigate how the sharpened bound depends on $\kappa_i$ as well as the spectral gap (\ref{rq:subsidiary:dependance}). Subsequently, we can simulate the eigenspectrum of a Darcy problem and compare the sharpened bound with the classical bound (\ref{rq:subsidiary:performance}). This will lead to a clear understanding of the performance of the sharpened bound in the main problem context of this thesis: heterogeneous scalar elliptic problems with high-contrast coefficient.

Furthermore, we can construct, discretize, and precondition a model Darcy problem with the methods outlined in \cref{sec:tailored_coarse_spaces}. Then, we apply both the sharpened bound and the CG method to the resulting systems, and investigate how sharp the new bound is (\ref{rq:subsidiary:preconditioners}).

The main challenge described in \cref{sec:challenges} still stands. More work is needed to be able to use the sharpened bound for spectra that are not known or artificially constructed beforehand. The results in this chapter suggest that the cluster condition number is a good candidate for a measure of the eigenspectrum. However, it is not yet clear how to estimate the cluster condition number for a general eigenspectrum. This is an important step towards answering \ref{rq:subsidiary:estimation}.