\chapter{Conclusion}\label{ch:conclusion}
This thesis stresses that the classical condition number-based CG iteration bound $m_1$ from \cref{eq:cg_convergence_rate_bound_iterations_approx} does not fully capture the convergence behavior in high-contrast heterogeneous elliptic problems, particularly when two-level Schwarz preconditioners are employed. The spectra of the preconditioned systems $\sigma(M^{-1}A), \ \forall M^{-1}\in\mathcal{M}^{-1}$ with $\mathcal{M}^{-1}$ as in \cref{eq:preconditioners}, were found in \cite{ams_coarse_space_comp_study_Alves2024} to not only have a condition number like that of \cref{eq:msfem_condition_number}, but also exhibit the spectral gap discussed in \cref{sec:spectral_gap_darcy}. The presence of this spectral gap suggests that the eigenspectrum of the preconditioned system is not uniformly distributed, which is a key assumption in deriving the classical CG iteration bound. This leads to the main motivation of this thesis that the classical bound is not sharp enough for high-contrast problems.

A study of the relevant literature in \cref{sec:cg_nonuniform_spectra} reveals that a two-cluster sharpened CG iterations bound was derived in \cite[Section 4]{cg_sharpened_convrate_Axelsson1976} in terms of the four extremal eigenvalues of the two clusters in a given spectrum. In \cref{sec:cg_sharpened_convrate}, this bound is derived once again. Necessary conditions \cref{eq:threshold_inequality_explicit_expansion,eq:tail_cluster_bound_sparsity_condition} for which $m_2 < m_1$ are derived in \cref{sec:performance_ratio,sec:tail_cluster_bound_performance} and in the process we obtain the three spectral characteristics that \cref{rq:subsidiary:measures} seeks to identity: the left- and right cluster condition numbers $\kappa_l, \kappa_r$ as well as the spectral width $s = \frac{\kappa}{\kappa_l}$.

Using the aforementioned necessary conditions based on $\kappa_l,\kappa_r$ and $s$ ensuring $m_2 < m_1$ in combination with a recursive partitioning algorithm and a generalized multi-cluster version of the bound from \cite{cg_sharpened_convrate_Axelsson1976}, we obtain in \cref{sec:cg_iteration_bound_algorithm} two new sharpened CG iteration bounds: a multi-cluster bound \cref{alg:multi_cluster_cg_bound}, $m_{N_{\text{cluster}}}$, and a tail-cluster bound \cref{alg:multi_tail_cluster_cg_bound}, $m_{N_{\text{tail-cluster}}}$. 

The sharpness of these bounds is investigated in \cref{sec:sharpness_of_bounds} by applying them to approximate eigenspectra obtained from the Lanczos matrix, resulting from a converged PCG method applied to a set of preconditioned, systems from discretized high-contrast heterogeneous elliptic problems as in \cref{prob:elliptic_problem_discretized}. In all cases the tail-cluster bound $m_{N_{\text{tail-cluster}}}$ outperforms the multi-cluster bound $m_{N_{\text{cluster}}}$, which in turn outperforms the classical condition number-based bound $m_1$, answering \cref{rq:subsidiary:heuristic} positively.

In \cref{sec:early_estimation_of_iterations} we show that though the new bounds $m_{N_{\text{cluster}}}$ and $m_{N_{\text{tail-cluster}}}$ are sharper than the classical bound $m_1$, they are not always applicable as they may require more information about the spectrum than is available a priori or computationally feasible to obtain by doing the first few PCG iterations. The culprit being in this case that rate of convergence of the Ritz values to the true eigenvalues is not fast enough to obtain a good estimate of the spectrum. Hence, the answer to \cref{rq:subsidiary:preconditioners} is that it depends on the specific coefficient function and preconditioner used.

In conclusion, the main goal of this thesis was to sharpen the CG iteration bound for Schwarz-preconditioned high-contrast heterogeneous scalar-elliptic problems beyond the classical condition number-based bound. This was achieved by deriving a two-cluster sharpened CG iteration bound and a tail-cluster sharpened CG iteration bound, which outperform the classical bound in terms of sharpness. The results show that the spectral characteristics of the preconditioned system play a crucial role in determining the convergence behavior of the (P)CG method, and that these characteristics can be used to obtain sharper bounds than the classical condition number-based bound. Further research is needed to explore the applicability of these bounds in practice and to develop methods for obtaining a priori estimates of the spectral characteristics necessary for their application.