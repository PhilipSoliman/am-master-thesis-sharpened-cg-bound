\chapter{Implementation}\label{ch:implementation}
In this chapter we discuss the implementation of \cref{prob:elliptic_problem_discretized} and the PCG method used to solve it, together with the relevant preconditioners from \cref{sec:schwarz_methods,sec:tailored_coarse_spaces}. We narrow the scope of the family of problems described by \cref{prob:elliptic_problem_discretized} to three specific instances of the coefficient function $\mathcal{C}$. All implementations in this chapter, as well as the algorithms described in \cref{sec:cg_multi_cluster_bound_algorithm,sec:cg_iteration_bound_algorithm} are available in the online \href{https://github.com/PhilipSoliman/hcmsfem}{\textit{High-Contrast Multi-Scale FEM}} repository. For specific references to scripts or classes in the repository, we refer to the repository's top-level \texttt{README.md} file.

\section{Implementation of the elliptic problem}\label{sec:implementation_elliptic_problem}
We consider a square domain $\Omega = [0,1]^2$ and introduce two conforming quadrilateral meshes $Q_h$ and $Q_H$ with fine and coarse mesh sizes $h$ and $H$, respectively, where $h = H/2^r$ and $r\in\mathbb{N}$ a positive integer. We fix $r=4$ and limit our study to the set of meshes
\begin{equation}
    \mathcal{Q} = \{(Q_h, Q_H)\mid H\in\{1/4, 1/8, 1/16, 1/32, 1/64\}\}.
    \label{eq:meshes}
\end{equation}
The \textit{fine} and \textit{coarse} meshes $Q_h, Q_H$ for $H=1/4$ are shown in \cref{fig:mesh_4}.
\begin{figure}[H]
    \centering
    \includegraphics{mesh_and_coefficient_functions_meshes_and_domains.pdf}
    \caption{Plot of the conforming fine and coarse meshes $Q_h$ and $Q_H$ for $H=1/4$ (\textbf{left}) and some of the overlapping subdomains $\Omega_i$ (\textbf{right}).}
    \label{fig:mesh_4}
\end{figure}
Next, we construct a FE space $V_h$ from first order Legendre polynomials $\phi_i$ associated to each internal fine mesh vertex $v^h_i$ in $Q_h$ and locally defined on all quadrilateral elements $q_j$ sharing $v^h_i$. That is
\[
    \text{supp}(\phi_i) = \bigcup_{j: v^h_i\in q_j, q_j \in Q_h} q_j, \quad \forall i\in\{1,\ldots,n\}.
\]
This defines the stiffness matrix $A$ and load vector $\mathbf{b}$ as specified in \cref{prob:elliptic_problem_discretized}.

We construct for each fine mesh $Q_h$ three coefficient functions $\mathcal{C}_{\text{const}}\equiv 1$, $\mathcal{C}_{\text{3layer, vert}}$ and $\mathcal{C}_{\text{edge slabs, around vertices}}$, the latter two of which are high-contrast coefficient functions with a periodic structure, as shown in \cref{fig:coefficient_functions}. 
\begin{figure}[H]
    \includegraphics{mesh_and_coefficient_functions_coefficient_functions.pdf}
    \caption{Plot of the coefficient function $\mathcal{C}_{\text{3layer, vert}}$ and $\mathcal{C}_{\text{edge slabs, around vertices}}$ defined on the fine mesh $Q_h$ for $H=1/4$. The contrast is $\frac{\mathcal{C}_{\text{max}}}{\mathcal{C}_{\text{min}}} = 10^8$ for both coefficient functions.}
    \label{fig:coefficient_functions}
\end{figure}
The coefficient functions $\mathcal{C}_{\text{3layer, vert}}$, $\mathcal{C}_{\text{edge slabs, around vertices}}$ are centered on or around those fine mesh vertices that lie on two coarse mesh edges $e_i^H\in Q_H$. The periodic structure of these coefficient functions is replicated when more subdomains are added. That is, the number of inclusions is kept proportional to the number of subdomains.

Finally, it is important to note that the meshes $\mathcal{M}$, finite element space $V_h$ and coefficient functions in \cref{fig:coefficient_functions} are the same as the ones in \cite{ams_coarse_space_comp_study_Alves2024}.

\section{Implementation of the PCG method}\label{sec:implementation_pcg}
We implement a \ref{pcg_type:left} method to solve the linear system arising from the discretization of the elliptic problem. To that end we first decompose the fine mesh $Q_h$ into overlapping subdomains $\Omega_i$ to be used for the alternating Schwarz method, as visualized in \cref{fig:mesh_4}. Then, we construct preconditioners with a general form similar to \cref{eq:ASM_preconditioner_coarse}
\[
    M^{-1} = R_0^T A_0^{-1} R_0 + \sum_{i=1}^{N_{\text{sub}}} R_i^T A_i^{-1} R_i,
\]
where $R_i$ is the \textit{restriction operator} to and $A_i = R_i^T A R_i$ is the \textit{local operator} on $\Omega_i \ \forall i \geq 1$. Also, $N_{\text{sub}} = (1/H)^2$, or simply the number of coarse mesh elements in $Q_H$. Similarly, the \textit{coarse restriction} operator $R_0$ and corresponding \textit{coarse operator} $A_0 = R_0^T A R_0$ are constructed as discussed in \cref{sec:tailored_coarse_spaces}. We construct $R_0$ for the GDSW, RGDSW, and AMS coarse spaces, resulting in the following two-level overlapping Schwarz preconditioners
\begin{equation}
    \mathcal{M}^{-1} =  \left\{M^{-1}_{\text{2-OAS-GDSW}}, M^{-1}_{\text{2-OAS-RGDSW}}, M^{-1}_{\text{2-OAS-AMS}}\right\}.
    \label{eq:preconditioners}
\end{equation}