\chapter{Results}\label{ch:results}

\section{Application to eigenspectra at convergence}
\begin{figure}[H]
    \centering
    \includegraphics[width=\textwidth]{absolute_performance.pdf}
    \caption{Plot of the number of CG iterations $m$ required to achieve convergence of the solution to \cref{prob:elliptic_problem_discretized} in the sense of criterion \cref{eq:residual_convergence_criterion_r} with $\epsilon_r=10^{-8}$. The left and right columns correspons to the $\mathcal{C}_{\text{3layer, vert}}$ and $\mathcal{C}_{\text{edge slabs, around vertices}}$ coefficient functions. Also shown are the corresponding classical $m_1$, multi-cluster $m_{N_{\text{cluster}}}$, and multi-tail-cluster $m_{N_{\text{tail-cluster}}}$ bounds for the CG method applied to the eigenspectra obtained from the Lanczos matrix convergence (Ritz values).}
    \label{fig:cg_sharpened_bound}
\end{figure}

\section{Early estimation of CG iterations}
\begin{table}[H]
\centering
\caption{PCG iteration bounds for solving the model diffusion problem with coefficient function $\mathcal{C}_{\mathrm{const}}$. Bounds are based on approximate spectra (Ritz values) obtained during the initial PCG iterations and are show for meshes $H=1/4$, $H=1/8$, $H=1/16$, $H=1/32$, $H=1/64$ and 2-OAS preconditioners with GDSW, RGDSW, AMS coarse spaces. The $\textbf{bound}$ columns show the values of the CG iteration bounds $m_1$, $m_{N_{\text{cluster}}}$, $m_{N_{\text{tail-cluster}}}$, $m_{\text{estimate}}$ and the $\textbf{iter.}$ columns show the iteration at which those bounds are obtained.}
\label{tab:cg_iteration_bound_coef=const}
\begin{tabular}{llrrrrrr}
\toprule
 &  & ('$m$',) & ('$m_1$',) & ('$m_{N_{\\text{cluster}}}$',) & ('$m_{N_{\\text{tail-cluster}}}$',) & ('$m_{\\text{estimate}}$',) & ('iter.',) \\
\midrule
\multirow[c]{3}{*}{\bfseries $H=1/4$} & \bfseries GDSW & 23 & {\cellcolor[HTML]{91A1F0}} \color[HTML]{F1F1F1} 34 & {\cellcolor[HTML]{91A1F0}} \color[HTML]{F1F1F1} 34 & {\cellcolor[HTML]{E2E4FB}} \color[HTML]{000000} 11 & {\cellcolor[HTML]{405FE5}} \color[HTML]{F1F1F1} 23 & 11 \\
\cline{2-8}
\bfseries  & \bfseries RGDSW & 21 & {\cellcolor[HTML]{E2E4FB}} \color[HTML]{000000} 40 & {\cellcolor[HTML]{E2E4FB}} \color[HTML]{000000} 40 & {\cellcolor[HTML]{91A1F0}} \color[HTML]{F1F1F1} 11 & {\cellcolor[HTML]{405FE5}} \color[HTML]{F1F1F1} 26 & 11 \\
\cline{2-8}
\bfseries  & \bfseries AMS & 18 & {\cellcolor[HTML]{91A1F0}} \color[HTML]{F1F1F1} 23 & {\cellcolor[HTML]{91A1F0}} \color[HTML]{F1F1F1} 23 & {\cellcolor[HTML]{E2E4FB}} \color[HTML]{000000} 11 & {\cellcolor[HTML]{405FE5}} \color[HTML]{F1F1F1} 17 & 11 \\
\cline{1-8} \cline{2-8}
\multirow[c]{3}{*}{\bfseries $H=1/8$} & \bfseries GDSW & 30 & {\cellcolor[HTML]{91A1F0}} \color[HTML]{F1F1F1} 37 & {\cellcolor[HTML]{91A1F0}} \color[HTML]{F1F1F1} 37 & {\cellcolor[HTML]{E2E4FB}} \color[HTML]{000000} 16 & {\cellcolor[HTML]{405FE5}} \color[HTML]{F1F1F1} 27 & 16 \\
\cline{2-8}
\bfseries  & \bfseries RGDSW & 32 & {\cellcolor[HTML]{91A1F0}} \color[HTML]{F1F1F1} 43 & {\cellcolor[HTML]{91A1F0}} \color[HTML]{F1F1F1} 43 & {\cellcolor[HTML]{E2E4FB}} \color[HTML]{000000} 11 & {\cellcolor[HTML]{405FE5}} \color[HTML]{F1F1F1} 27 & 11 \\
\cline{2-8}
\bfseries  & \bfseries AMS & 21 & {\cellcolor[HTML]{405FE5}} \color[HTML]{F1F1F1} 23 & {\cellcolor[HTML]{405FE5}} \color[HTML]{F1F1F1} 23 & {\cellcolor[HTML]{E2E4FB}} \color[HTML]{000000} 11 & {\cellcolor[HTML]{91A1F0}} \color[HTML]{F1F1F1} 17 & 11 \\
\cline{1-8} \cline{2-8}
\multirow[c]{3}{*}{\bfseries $H=1/16$} & \bfseries GDSW & 33 & {\cellcolor[HTML]{405FE5}} \color[HTML]{F1F1F1} 37 & {\cellcolor[HTML]{405FE5}} \color[HTML]{F1F1F1} 37 & {\cellcolor[HTML]{E2E4FB}} \color[HTML]{000000} 16 & {\cellcolor[HTML]{91A1F0}} \color[HTML]{F1F1F1} 27 & 16 \\
\cline{2-8}
\bfseries  & \bfseries RGDSW & 38 & {\cellcolor[HTML]{405FE5}} \color[HTML]{F1F1F1} 45 & {\cellcolor[HTML]{405FE5}} \color[HTML]{F1F1F1} 45 & {\cellcolor[HTML]{E2E4FB}} \color[HTML]{000000} 16 & {\cellcolor[HTML]{405FE5}} \color[HTML]{F1F1F1} 31 & 16 \\
\cline{2-8}
\bfseries  & \bfseries AMS & 22 & {\cellcolor[HTML]{405FE5}} \color[HTML]{F1F1F1} 24 & {\cellcolor[HTML]{405FE5}} \color[HTML]{F1F1F1} 24 & {\cellcolor[HTML]{E2E4FB}} \color[HTML]{000000} 16 & {\cellcolor[HTML]{405FE5}} \color[HTML]{F1F1F1} 20 & 16 \\
\cline{1-8} \cline{2-8}
\multirow[c]{3}{*}{\bfseries $H=1/32$} & \bfseries GDSW & 35 & {\cellcolor[HTML]{405FE5}} \color[HTML]{F1F1F1} 37 & {\cellcolor[HTML]{405FE5}} \color[HTML]{F1F1F1} 37 & {\cellcolor[HTML]{E2E4FB}} \color[HTML]{000000} 16 & {\cellcolor[HTML]{91A1F0}} \color[HTML]{F1F1F1} 27 & 16 \\
\cline{2-8}
\bfseries  & \bfseries RGDSW & 42 & {\cellcolor[HTML]{405FE5}} \color[HTML]{F1F1F1} 44 & {\cellcolor[HTML]{405FE5}} \color[HTML]{F1F1F1} 44 & {\cellcolor[HTML]{E2E4FB}} \color[HTML]{000000} 16 & {\cellcolor[HTML]{91A1F0}} \color[HTML]{F1F1F1} 30 & 16 \\
\cline{2-8}
\bfseries  & \bfseries AMS & 22 & {\cellcolor[HTML]{405FE5}} \color[HTML]{F1F1F1} 24 & {\cellcolor[HTML]{405FE5}} \color[HTML]{F1F1F1} 24 & {\cellcolor[HTML]{E2E4FB}} \color[HTML]{000000} 16 & {\cellcolor[HTML]{405FE5}} \color[HTML]{F1F1F1} 20 & 16 \\
\cline{1-8} \cline{2-8}
\multirow[c]{3}{*}{\bfseries $H=1/64$} & \bfseries GDSW & 36 & {\cellcolor[HTML]{405FE5}} \color[HTML]{F1F1F1} 37 & {\cellcolor[HTML]{405FE5}} \color[HTML]{F1F1F1} 37 & {\cellcolor[HTML]{E2E4FB}} \color[HTML]{000000} 16 & {\cellcolor[HTML]{91A1F0}} \color[HTML]{F1F1F1} 27 & 16 \\
\cline{2-8}
\bfseries  & \bfseries RGDSW & 45 & {\cellcolor[HTML]{405FE5}} \color[HTML]{F1F1F1} 44 & {\cellcolor[HTML]{405FE5}} \color[HTML]{F1F1F1} 44 & {\cellcolor[HTML]{E2E4FB}} \color[HTML]{000000} 16 & {\cellcolor[HTML]{91A1F0}} \color[HTML]{F1F1F1} 30 & 16 \\
\cline{2-8}
\bfseries  & \bfseries AMS & 23 & {\cellcolor[HTML]{91A1F0}} \color[HTML]{F1F1F1} 29 & {\cellcolor[HTML]{91A1F0}} \color[HTML]{F1F1F1} 29 & {\cellcolor[HTML]{E2E4FB}} \color[HTML]{000000} 11 & {\cellcolor[HTML]{405FE5}} \color[HTML]{F1F1F1} 20 & 11 \\
\cline{1-8} \cline{2-8}
\bottomrule
\end{tabular}
\end{table}

\begin{table}[H]
\centering
\caption{PCG iteration bounds $m_1$, $m_{N_{\text{cluster}}}$, $m_{N_{\text{tail-cluster}}}$, $m_{\text{estimate}}$ for solving the model diffusion problem with coefficient function $\mathcal{C}_{\mathrm{3layer, \ vert}}$. Bounds are based on approximate spectra (Ritz values) obtained during the initial PCG iterations and are shown for meshes $\mathbf{H=1/4}$, $\mathbf{H=1/8}$, $\mathbf{H=1/16}$, $\mathbf{H=1/32}$, $\mathbf{H=1/64}$ and 2-OAS preconditioners with GDSW, RGDSW, AMS coarse spaces. The $\textbf{iter.}$ column shows the iteration at which the bounds are obtained.}
\label{tab:cg_iteration_bound_coef=3lvert}
\begin{tabular}{llrrrrrr}
\toprule
 &  & $m$ & $m_1$ & $m_{N_{\text{cluster}}}$ & $m_{N_{\text{tail-cluster}}}$ & $m_{\text{estimate}}$ & \textbf{iter.} \\
\midrule
\multirow[c]{3}{*}{$\mathbf{H=1/4}$} & GDSW & 62 & {\cellcolor[HTML]{9CC3F5}} \color[HTML]{000000} 225,419 & {\cellcolor[HTML]{999EC0}} \color[HTML]{F1F1F1} 111 & {\cellcolor[HTML]{97788C}} \color[HTML]{F1F1F1} 26 & {\cellcolor[HTML]{945357}} \color[HTML]{F1F1F1} 69 & 26 \\
\cline{2-8}
 & RGDSW & 78 & {\cellcolor[HTML]{9CC3F5}} \color[HTML]{000000} 224,028 & {\cellcolor[HTML]{97788C}} \color[HTML]{F1F1F1} 109 & {\cellcolor[HTML]{999EC0}} \color[HTML]{F1F1F1} 26 & {\cellcolor[HTML]{945357}} \color[HTML]{F1F1F1} 68 & 26 \\
\cline{2-8}
 & AMS & 20 & {\cellcolor[HTML]{945357}} \color[HTML]{F1F1F1} 23 & {\cellcolor[HTML]{945357}} \color[HTML]{F1F1F1} 23 & {\cellcolor[HTML]{9CC3F5}} \color[HTML]{000000} 11 & {\cellcolor[HTML]{945357}} \color[HTML]{F1F1F1} 17 & 11 \\
\cline{1-8} \cline{2-8}
\multirow[c]{3}{*}{$\mathbf{H=1/8}$} & GDSW & 237 & {\cellcolor[HTML]{9CC3F5}} \color[HTML]{000000} 300,788 & {\cellcolor[HTML]{999EC0}} \color[HTML]{F1F1F1} 641 & {\cellcolor[HTML]{97788C}} \color[HTML]{F1F1F1} 91 & {\cellcolor[HTML]{945357}} \color[HTML]{F1F1F1} 366 & 51 \\
\cline{2-8}
 & RGDSW & 242 & {\cellcolor[HTML]{9CC3F5}} \color[HTML]{000000} 302,967 & {\cellcolor[HTML]{999EC0}} \color[HTML]{F1F1F1} 638 & {\cellcolor[HTML]{97788C}} \color[HTML]{F1F1F1} 86 & {\cellcolor[HTML]{945357}} \color[HTML]{F1F1F1} 362 & 51 \\
\cline{2-8}
 & AMS & 22 & {\cellcolor[HTML]{945357}} \color[HTML]{F1F1F1} 23 & {\cellcolor[HTML]{945357}} \color[HTML]{F1F1F1} 23 & {\cellcolor[HTML]{9CC3F5}} \color[HTML]{000000} 11 & {\cellcolor[HTML]{988BA6}} \color[HTML]{F1F1F1} 17 & 11 \\
\cline{1-8} \cline{2-8}
\multirow[c]{3}{*}{$\mathbf{H=1/16}$} & GDSW & 403 & {\cellcolor[HTML]{9CC3F5}} \color[HTML]{000000} 350,992 & {\cellcolor[HTML]{999EC0}} \color[HTML]{F1F1F1} 949 & {\cellcolor[HTML]{97788C}} \color[HTML]{F1F1F1} 148 & {\cellcolor[HTML]{945357}} \color[HTML]{F1F1F1} 549 & 96 \\
\cline{2-8}
 & RGDSW & 442 & {\cellcolor[HTML]{9CC3F5}} \color[HTML]{000000} 351,891 & {\cellcolor[HTML]{999EC0}} \color[HTML]{F1F1F1} 948 & {\cellcolor[HTML]{97788C}} \color[HTML]{F1F1F1} 147 & {\cellcolor[HTML]{945357}} \color[HTML]{F1F1F1} 548 & 96 \\
\cline{2-8}
 & AMS & 22 & {\cellcolor[HTML]{945357}} \color[HTML]{F1F1F1} 23 & {\cellcolor[HTML]{945357}} \color[HTML]{F1F1F1} 23 & {\cellcolor[HTML]{9CC3F5}} \color[HTML]{000000} 11 & {\cellcolor[HTML]{988BA6}} \color[HTML]{F1F1F1} 17 & 11 \\
\cline{1-8} \cline{2-8}
\multirow[c]{3}{*}{$\mathbf{H=1/32}$} & GDSW & 607 & {\cellcolor[HTML]{9CC3F5}} \color[HTML]{000000} 367,021 & {\cellcolor[HTML]{999EC0}} \color[HTML]{F1F1F1} 2,228 & {\cellcolor[HTML]{945357}} \color[HTML]{F1F1F1} 231 & {\cellcolor[HTML]{97788C}} \color[HTML]{F1F1F1} 1,230 & 161 \\
\cline{2-8}
 & RGDSW & 606 & {\cellcolor[HTML]{9CC3F5}} \color[HTML]{000000} 368,220 & {\cellcolor[HTML]{999EC0}} \color[HTML]{F1F1F1} 2,277 & {\cellcolor[HTML]{945357}} \color[HTML]{F1F1F1} 290 & {\cellcolor[HTML]{97788C}} \color[HTML]{F1F1F1} 1,284 & 216 \\
\cline{2-8}
 & AMS & 22 & {\cellcolor[HTML]{945357}} \color[HTML]{F1F1F1} 23 & {\cellcolor[HTML]{945357}} \color[HTML]{F1F1F1} 23 & {\cellcolor[HTML]{9CC3F5}} \color[HTML]{000000} 11 & {\cellcolor[HTML]{988BA6}} \color[HTML]{F1F1F1} 17 & 11 \\
\cline{1-8} \cline{2-8}
\multirow[c]{3}{*}{$\mathbf{H=1/64}$} & GDSW & 797 & {\cellcolor[HTML]{9CC3F5}} \color[HTML]{000000} 359,087 & {\cellcolor[HTML]{999EC0}} \color[HTML]{F1F1F1} 1,764 & {\cellcolor[HTML]{97788C}} \color[HTML]{F1F1F1} 85 & {\cellcolor[HTML]{945357}} \color[HTML]{F1F1F1} 925 & 46 \\
\cline{2-8}
 & RGDSW & 778 & {\cellcolor[HTML]{9CC3F5}} \color[HTML]{000000} 359,976 & {\cellcolor[HTML]{999EC0}} \color[HTML]{F1F1F1} 1,763 & {\cellcolor[HTML]{97788C}} \color[HTML]{F1F1F1} 85 & {\cellcolor[HTML]{945357}} \color[HTML]{F1F1F1} 924 & 51 \\
\cline{2-8}
 & AMS & 22 & {\cellcolor[HTML]{945357}} \color[HTML]{F1F1F1} 23 & {\cellcolor[HTML]{945357}} \color[HTML]{F1F1F1} 23 & {\cellcolor[HTML]{9CC3F5}} \color[HTML]{000000} 11 & {\cellcolor[HTML]{988BA6}} \color[HTML]{F1F1F1} 17 & 11 \\
\cline{1-8} \cline{2-8}
\bottomrule
\end{tabular}
\end{table}

\begin{table}[H]
\centering
\caption{PCG iteration bounds $m_1$, $m_{N_{\text{cluster}}}$, $m_{N_{\text{tail-cluster}}}$, $m_{\text{estimate}}$ for solving the model diffusion problem with coefficient function $\mathcal{C}_{\mathrm{edge \ slabs, \ around \ vertices}}$. Bounds are based on approximate spectra (Ritz values) obtained during the initial PCG iterations and are shown for meshes $\mathbf{H=1/4}$, $\mathbf{H=1/8}$, $\mathbf{H=1/16}$, $\mathbf{H=1/32}$, $\mathbf{H=1/64}$ and 2-OAS preconditioners with GDSW, RGDSW, AMS coarse spaces. The $i$ column shows the iteration at which the bounds are obtained. The color of each cell indicates whether the bound is larger (blue) or smaller (red) than the number of iterations required for convergence $m$. The shade of the cell is proportional to the absolute difference between $m$ and the bound.}
\label{tab:cg_iteration_bound_coef=slabs_around_vertices}
\begin{tabular}{llrrrrrr}
\toprule
 &  & $m$ & $m_1$ & $m_{N_{\text{cluster}}}$ & $m_{N_{\text{tail-cluster}}}$ & $m_{\text{estimate}}$ & $i$ \\
\midrule
\multirow[c]{3}{*}{$\mathbf{H=1/4}$} & GDSW & 80 & {\cellcolor[HTML]{AFC9F6}} \color[HTML]{000000} 124,727 & {\cellcolor[HTML]{7EAFF1}} \color[HTML]{000000} 88 & {\cellcolor[HTML]{BB9CA9}} \color[HTML]{F1F1F1} 36 & {\cellcolor[HTML]{945357}} \color[HTML]{F1F1F1} 62 & 36 \\
\cline{2-8}
 & RGDSW & 81 & {\cellcolor[HTML]{C1D2F8}} \color[HTML]{000000} 117,699 & {\cellcolor[HTML]{9FC0F4}} \color[HTML]{000000} 133 & {\cellcolor[HTML]{945357}} \color[HTML]{F1F1F1} 36 & {\cellcolor[HTML]{7EAFF1}} \color[HTML]{000000} 85 & 36 \\
\cline{2-8}
 & AMS & 83 & {\cellcolor[HTML]{C9D6F8}} \color[HTML]{000000} 98,421 & {\cellcolor[HTML]{AFC9F6}} \color[HTML]{000000} 191 & {\cellcolor[HTML]{7EAFF1}} \color[HTML]{000000} 84 & {\cellcolor[HTML]{96BCF3}} \color[HTML]{000000} 138 & 41 \\
\cline{1-8} \cline{2-8}
\multirow[c]{3}{*}{$\mathbf{H=1/8}$} & GDSW & 261 & {\cellcolor[HTML]{C1D2F8}} \color[HTML]{000000} 127,648 & {\cellcolor[HTML]{9FC0F4}} \color[HTML]{000000} 383 & {\cellcolor[HTML]{945357}} \color[HTML]{F1F1F1} 173 & {\cellcolor[HTML]{7EAFF1}} \color[HTML]{000000} 278 & 96 \\
\cline{2-8}
 & RGDSW & 494 & {\cellcolor[HTML]{C9D6F8}} \color[HTML]{000000} 124,745 & {\cellcolor[HTML]{AFC9F6}} \color[HTML]{000000} 1,274 & {\cellcolor[HTML]{7EAFF1}} \color[HTML]{000000} 583 & {\cellcolor[HTML]{96BCF3}} \color[HTML]{000000} 929 & 186 \\
\cline{2-8}
 & AMS & 171 & {\cellcolor[HTML]{C1D2F8}} \color[HTML]{000000} 110,677 & {\cellcolor[HTML]{9FC0F4}} \color[HTML]{000000} 302 & {\cellcolor[HTML]{945357}} \color[HTML]{F1F1F1} 123 & {\cellcolor[HTML]{7EAFF1}} \color[HTML]{000000} 213 & 71 \\
\cline{1-8} \cline{2-8}
\multirow[c]{3}{*}{$\mathbf{H=1/16}$} & GDSW & 346 & {\cellcolor[HTML]{7EAFF1}} \color[HTML]{000000} 123,872 & {\cellcolor[HTML]{945357}} \color[HTML]{F1F1F1} 306 & {\cellcolor[HTML]{C8B4C4}} \color[HTML]{000000} 134 & {\cellcolor[HTML]{AE838D}} \color[HTML]{F1F1F1} 220 & 76 \\
\cline{2-8}
 & RGDSW & 1,406 & {\cellcolor[HTML]{C1D2F8}} \color[HTML]{000000} 122,195 & {\cellcolor[HTML]{9FC0F4}} \color[HTML]{000000} 2,185 & {\cellcolor[HTML]{945357}} \color[HTML]{F1F1F1} 979 & {\cellcolor[HTML]{7EAFF1}} \color[HTML]{000000} 1,582 & 241 \\
\cline{2-8}
 & AMS & 238 & {\cellcolor[HTML]{AFC9F6}} \color[HTML]{000000} 110,969 & {\cellcolor[HTML]{7EAFF1}} \color[HTML]{000000} 310 & {\cellcolor[HTML]{BB9CA9}} \color[HTML]{F1F1F1} 126 & {\cellcolor[HTML]{945357}} \color[HTML]{F1F1F1} 218 & 81 \\
\cline{1-8} \cline{2-8}
\multirow[c]{3}{*}{$\mathbf{H=1/32}$} & GDSW & 363 & {\cellcolor[HTML]{7EAFF1}} \color[HTML]{000000} 122,634 & {\cellcolor[HTML]{945357}} \color[HTML]{F1F1F1} 310 & {\cellcolor[HTML]{C8B4C4}} \color[HTML]{000000} 136 & {\cellcolor[HTML]{AE838D}} \color[HTML]{F1F1F1} 223 & 76 \\
\cline{2-8}
 & RGDSW & 3,082 & {\cellcolor[HTML]{7EAFF1}} \color[HTML]{000000} 109,007 & {\cellcolor[HTML]{945357}} \color[HTML]{F1F1F1} 1,415 & {\cellcolor[HTML]{C8B4C4}} \color[HTML]{000000} 920 & {\cellcolor[HTML]{AE838D}} \color[HTML]{F1F1F1} 1,168 & 271 \\
\cline{2-8}
 & AMS & 276 & {\cellcolor[HTML]{AFC9F6}} \color[HTML]{000000} 110,970 & {\cellcolor[HTML]{7EAFF1}} \color[HTML]{000000} 312 & {\cellcolor[HTML]{BB9CA9}} \color[HTML]{F1F1F1} 127 & {\cellcolor[HTML]{945357}} \color[HTML]{F1F1F1} 220 & 86 \\
\cline{1-8} \cline{2-8}
\multirow[c]{3}{*}{$\mathbf{H=1/64}$} & GDSW & 407 & {\cellcolor[HTML]{AFC9F6}} \color[HTML]{000000} 127,897 & {\cellcolor[HTML]{7EAFF1}} \color[HTML]{000000} 485 & {\cellcolor[HTML]{BB9CA9}} \color[HTML]{F1F1F1} 268 & {\cellcolor[HTML]{945357}} \color[HTML]{F1F1F1} 377 & 186 \\
\cline{2-8}
 & RGDSW & 6,766 & {\cellcolor[HTML]{7EAFF1}} \color[HTML]{000000} 69,645 & {\cellcolor[HTML]{945357}} \color[HTML]{F1F1F1} 2,612 & {\cellcolor[HTML]{C8B4C4}} \color[HTML]{000000} 956 & {\cellcolor[HTML]{AE838D}} \color[HTML]{F1F1F1} 1,784 & 291 \\
\cline{2-8}
 & AMS & 310 & {\cellcolor[HTML]{AFC9F6}} \color[HTML]{000000} 114,629 & {\cellcolor[HTML]{7EAFF1}} \color[HTML]{000000} 324 & {\cellcolor[HTML]{BB9CA9}} \color[HTML]{F1F1F1} 132 & {\cellcolor[HTML]{945357}} \color[HTML]{F1F1F1} 228 & 91 \\
\cline{1-8} \cline{2-8}
\bottomrule
\end{tabular}
\end{table}


\section{Implications for research}\label{sec:cg_sharpened_convrate_implications}
The preliminary results discussed in this chapter show that we can find both an a priori analytic two-cluster (\cref{eq:cg_iteration_bound_2_clusters}) and multiple-cluster (\cref{eq:cg_iteration_bound_multiple_clusters}) sharpened iteration bound for the CG method. The latter can also be implemented as a sequential, numerical algorithm that can be applied to artificially constructed spectra in \cref{fig:cg_sharpened_bound}. The sharpened bound appears to perform best in the two-cluster case, which has the most correspondence to the eigenspectrum of a typical (preconditioned) Darcy problem. Hence, \ref{rq:subsidiary:heuristic} is answered positively. 

With regard to \ref{rq:subsidiary:measures}, the cluster condition number $\kappa_i$ is introduced as a measure of the cluster width. This suggests that we can use the cluster condition number to distinguish between different preconditioners which is a promising result for \ref{rq:subsidiary:preconditioners}. Moreover, inspecting \cref{eq:cg_iteration_bound_2_clusters} more closely for the eigenspectrum in \cref{eq:two_clusters} (case $i=1$) reveals the presence of a sort of spectral gap $\frac{d}{b}$. This serves as yet another candidate for a potential set of spectral characteristics to estimate the distribution of eigenvalues in the eigenspectrum.

A logical next step is to more rigorously investigate how the sharpened bound depends on $\kappa_i$ as well as the spectral gap (\ref{rq:subsidiary:dependance}). Subsequently, we can simulate the eigenspectrum of a Darcy problem and compare the sharpened bound with the classical bound (\ref{rq:subsidiary:performance}). This will lead to a clear understanding of the performance of the sharpened bound in the main problem context of this thesis: heterogeneous scalar elliptic problems with high-contrast coefficient.

Furthermore, we can construct, discretize, and precondition a model Darcy problem with the methods outlined in \cref{sec:tailored_coarse_spaces}. Then, we apply both the sharpened bound and the CG method to the resulting systems, and investigate how sharp the new bound is (\ref{rq:subsidiary:preconditioners}).

The main challenge described in \cref{sec:challenges} still stands. More work is needed to be able to use the sharpened bound for spectra that are not known or artificially constructed beforehand. The results in this chapter suggest that the cluster condition number is a good candidate for a measure of the eigenspectrum. However, it is not yet clear how to estimate the cluster condition number for a general eigenspectrum. This is an important step towards answering \ref{rq:subsidiary:estimation}.