\pagestyle{empty}

\section*{Symbols}
% Symbols for the Heterogeneous Elliptic Problem
\begin{longtable}{c p{10cm}}
    \caption{Symbols related to the heterogeneous elliptic problem.}\label{tab:elliptic_symbols}                     \\
    \hline
    \textbf{Symbol}        & \textbf{Description}                                                                    \\
    \hline
    \endfirsthead

    \hline
    \textbf{Symbol}        & \textbf{Description}                                                                    \\
    \hline
    \endhead

    \hline
    \endfoot

    \hline
    \endlastfoot

    $\Omega$               & Bounded domain in $\mathbb{R}^d$ with Lipschitz boundary.                               \\
    $\partial \Omega$      & Boundary of $\Omega$.                                                                   \\
    $\mathbb{R}^d$         & $d$-dimensional Euclidean space.                                                        \\
    $\mathcal{C}$          & Scalar coefficient field in the elliptic problem, assumed to lie in $L^\infty(\Omega)$. \\
    $\mathcal{C}_{\min}$   & Lower bound of the scalar coefficient $\mathcal{C}$.                                    \\
    $\mathcal{C}_{\max}$   & Upper bound of the scalar coefficient $\mathcal{C}$.                                    \\
    $u$                    & Exact solution of the elliptic problem.                                                 \\
    $f$                    & Source term belonging to $L^2(\Omega)$.                                                 \\
    $u_D$                  & Dirichlet boundary data.                                                                \\
    $u_h$                  & Finite element approximation of $u$.                                                    \\
    $V_h$                  & Finite-dimensional subspace of $H_0^1(\Omega)$.                                         \\
    $V$                    & Solution space $\{u\in H^1(\Omega) | u_{\delta \Omega} = u_D\}$.                        \\
    $V_{h,0}$              & Subspace $V_h\cap H^1_0(\Omega)$ with homogeneous boundary conditions.                  \\
    $H^1_0(\Omega)$        & Sobolev space of functions with zero trace on the boundary.                             \\
    $\{\phi_i\}_{i=1}^{n}$ & Basis functions spanning $V_h$.                                                         \\
    $\mathcal{T}$          & Triangulation of the domain $\Omega$.                                                   \\
    $\mathcal{N}$          & Set of degrees of freedom (DOFs).                                                       \\
    $n$                    & Number of degrees of freedom, $n = |\mathcal{N}|$.                                      \\
    $a(u_h, v_h)$          & Bilinear form $\int_\Omega \mathcal{C}\nabla u_h\cdot\nabla v_h\,dx$.                   \\
    $(f, v_h)$             & Linear form $\int_\Omega f v_h \,dx$.                                                   \\
    $A$                    & Stiffness matrix derived from the Galerkin method.                                      \\
    $\mathbf{b}$           & Load vector in the resulting linear system.                                             \\
    $\mathbf{u}$           & Solution vector with components $u_i$.                                                  \\
    $v_h$                  & Test function in $V_h$.                                                                 \\
    $\|\cdot\|_A$          & $A$-norm defined by $\|x\|_A = \sqrt{x^T A x}$.                                         \\
\end{longtable}

% Symbols for the Conjugate Gradient Method
\begin{longtable}{c p{10cm}}
    \caption{Symbols related to the Conjugate Gradient (CG) method.}\label{tab:cg_symbols}                                                         \\
    \hline
    \textbf{Symbol}                  & \textbf{Description}                                                                                        \\
    \hline
    \endfirsthead

    \hline
    \textbf{Symbol}                  & \textbf{Description}                                                                                        \\
    \hline
    \endhead

    \hline
    \endfoot

    \hline
    \endlastfoot

    $\mathbf{u}_0$                   & Initial guess for the solution.                                                                             \\
    $\mathbf{u}^*$                   & Exact solution of the linear system.                                                                        \\
    $\mathbf{u}_m$                   & Approximate solution at the $m^{\text{th}}$ iteration.                                                      \\
    $\mathbf{r}_0$                   & Initial residual defined as $\mathbf{b}-A\mathbf{u}_0$.                                                     \\
    $\mathbf{r}_j$                   & Residual vector at the $j^{\text{th}}$ iteration.                                                           \\
    $\mathbf{p}_j$                   & Search direction at iteration $j$.                                                                          \\
    $\alpha_j$                       & Step size computed at iteration $j$.                                                                        \\
    $\beta_j$                        & Coefficient used to update the search direction in iteration $j$.                                           \\
    $\mathcal{K}_m(A, \mathbf{r}_0)$ & Krylov subspace spanned by $\{\mathbf{r}_0, A\mathbf{r}_0, A^2\mathbf{r}_0, \ldots, A^{m-1}\mathbf{r}_0\}$. \\
    $\mathcal{K}_m$                  & Shorthand for Krylov subspace $\mathcal{K}_m(A, \mathbf{r}_0)$.                                             \\
    $\mathcal{L}$                    & Constraint subspace in projection methods.                                                                  \\
    $V$                              & Matrix whose columns span the subspace $\mathcal{K}$.                                                       \\
    $\mathbf{c}$                     & Correction vector in projection methods.                                                                    \\
    $H$                              & Hessenberg matrix defined as $H = V^TAV$.                                                                   \\
    $T_m$                            & Tridiagonal Hessenberg matrix arising from the Lanczos process.                                             \\
    $\mathbf{v}_j$                   & Lanczos vectors forming orthonormal basis for Krylov subspace.                                              \\
    $\delta_j$                       & Diagonal entries of $T_m$.                                                                                  \\
    $\eta_j$                         & Off-diagonal entries of $T_m$.                                                                              \\
    $\mathbf{e}_m$                   & Error at iteration $m$, defined as $\mathbf{u}^* - \mathbf{u}_m$.                                           \\
    $\mathcal{P}_{m-1}$              & Space of polynomials of degree at most $m-1$.                                                               \\
    $r_m(x)$                         & Residual polynomial of degree $m$ with $r_m(0) = 1$.                                                        \\
    $q_{m-1}(x)$                     & Solution polynomial of degree $m-1$ in CG.                                                                  \\
    $\mu$                            & Grade of a vector with respect to a matrix.                                                                 \\
    $\lambda_i$                      & Eigenvalues of $A$.                                                                                         \\
    $\lambda_{\min}$                 & Smallest eigenvalue of $A$.                                                                                 \\
    $\lambda_{\max}$                 & Largest eigenvalue of $A$.                                                                                  \\
    $\kappa(A)$                      & Condition number of $A$, defined as $\lambda_{\max}/\lambda_{\min}$.                                        \\
    $\xi_i$                          & Components of the initial error in the eigenvector basis of $A$.                                            \\
    $\sigma(A)$                      & Spectrum (set of eigenvalues) of $A$.                                                                       \\
    $C_m$                            & Chebyshev polynomial of degree $m$.                                                                         \\
    $\hat{C}_m$                      & Real-valued transformed Chebyshev polynomial adapted to interval $[a,b]$.                                   \\
    $\mathbf{\rho}_0$                & Initial residual in the eigenvector basis of $A$.                                                           \\
    $Q$                              & Orthonormal eigenbasis matrix in diagonalization $A = QDQ^T$.                                               \\
    $D$                              & Diagonal matrix of eigenvalues in diagonalization.                                                          \\
    $\epsilon$                       & Relative error tolerance.                                                                                   \\
    $\epsilon_r$                     & Relative residual tolerance.                                                                                \\
    $\epsilon_b$                     & Residual tolerance relative to right-hand side.                                                             \\
    $\tilde{\epsilon}$               & Absolute error tolerance.                                                                                   \\
    $m$                              & Number of CG iterations required for convergence.                                                           \\
    $m_1$                            & Classical CG iteration bound based on condition number.                                                     \\
    $f$                              & CG convergence rate $f = \frac{\sqrt{\kappa} - 1}{\sqrt{\kappa} + 1}$.                                      \\
\end{longtable}

% Symbols for the Schwarz Preconditioners
\begin{longtable}{c p{10cm}}
    \caption{Symbols related to Schwarz preconditioners.}\label{tab:schwarz_symbols}                                                                         \\
    \hline
    \textbf{Symbol}       & \textbf{Description}                                                                                                             \\
    \hline
    \endfirsthead

    \hline
    \textbf{Symbol}       & \textbf{Description}                                                                                                             \\
    \hline
    \endhead

    \hline
    \endfoot

    \hline
    \endlastfoot

    $\Omega_i$            & Subdomains obtained from partitioning $\Omega$.                                                                                  \\
    $R_i$                 & Restriction operator for subdomain $\Omega_i$.                                                                                   \\
    $D_i$                 & Diagonal matrix representing the partition of unity (weights) for $\Omega_i$.                                                    \\
    $N_{sub}$             & Number of subdomains.                                                                                                            \\
    $M$                   & Preconditioner matrix.                                                                                                           \\
    $M^{-1}_{\text{ASM}}$ & Additive Schwarz preconditioner defined by $M_{\text{ASM}} = \sum_{i=1}^{N_{sub}} R_i^T (R_i A R_i^T)^{-1} R_i$.                 \\
    $M^{-1}_{\text{RAS}}$ & Restrictive additive Schwarz preconditioner defined by $M_{\text{RAS}} = \sum_{i=1}^{N_{sub}} R_i^T D_i (R_i A R_i^T)^{-1} R_i$. \\
    $A_i$                 & Local operator on subdomain $\Omega_i$, defined as $A_i = R_i A R_i^T$.                                                          \\
    $R_0$                 & Restriction operator associated with the coarse space.                                                                           \\
    $A_0$                 & Coarse operator defined as $A_0 = R_0^T A R_0$.                                                                                  \\
    $P_j$                 & Local projection operator associated with subdomain $\Omega_j$.                                                                  \\
    $P_0$                 & Projection operator for the coarse space.                                                                                        \\
    $P_{ad}$              & Sum of the projection operators, $P_{ad} = \sum_{j=1}^{N_{sub}} P_j$, used in the two-level method.                              \\
    $\kappa(P_{ad})$      & Condition number of the preconditioned system, given by $\frac{\lambda_{\max}}{\lambda_{\min}}$ of $P_{ad}$.                     \\
    $C_0$                 & Constant in stability decomposition for coarse space analysis.                                                                   \\
    $k_0$                 & Maximum number of subdomains that overlap at any point.                                                                          \\
    $\mathcal{P}_j$       & $A$-symmetric projection operator for subdomain $\Omega_j$.                                                                      \\
    $E_j$                 & Extension operator from subdomain $\Omega_j$ to global domain.                                                                   \\
\end{longtable}

% Symbols related to eigenspectra and CG convergence bounds
\begin{longtable}{c p{10cm}}
    \caption{Symbols related to eigenspectra and CG convergence bounds \fullref{ch:methods}.}\label{tab:eigenspectra_symbols}                                                                    \\
    \hline
    \textbf{Symbol}                  & \textbf{Description}                                                                                                                                      \\
    \hline
    \endfirsthead

    \hline
    \textbf{Symbol}                  & \textbf{Description}                                                                                                                                      \\
    \hline
    \endhead

    \hline
    \endfoot

    \hline
    \endlastfoot

    $\sigma_1(A)$                    & Two-cluster eigenspectrum of $A$, defined as the union of two intervals $[a,b] \cup [c,d]$.                                                               \\
    $\sigma_2(A)$                    & Eigenspectrum comprising a main cluster and a tail of eigenvalues, with the tail denoted by $N_{\text{tail}}$.                                            \\
    $[a,b]$                          & Interval containing the first cluster of eigenvalues.                                                                                                     \\
    $[c,d]$                          & Interval containing the second cluster of eigenvalues.                                                                                                    \\
    $N_{\text{tail}}$                & Number of eigenvalues in the tail cluster (when the spectrum has one cluster plus a tail).                                                                \\
    $m_2$                            & Sharpened CG iteration bound for two-cluster eigenspectrum.                                                                                               \\
    $p$                              & Degree parameter in two-cluster CG bound.                                                                                                                 \\
    $\kappa_l$                       & Left cluster condition number $\kappa_l = b/a$.                                                                                                           \\
    $\kappa_r$                       & Right cluster condition number $\kappa_r = d/c$.                                                                                                          \\
    $\hat{r}^{(i)}_p(x)$             & Residual polynomial function for the left cluster, defined piecewise for the two-cluster ($i=1$) and tail-cluster ($i=2$) spectrum                        \\
    $\hat{r}_{\bar{m}-p}(x)$         & Residual polynomial function based on Chebyshev polynomials, corresponding to the complementary polynomial degree $\bar{m}-p$.                            \\
    $C^{(i)}_{p}$                    & Cluster-specific Chebyshev polynomial of degree $p$, adapted to the eigenvalue distribution of the $i^\text{th}$ cluster.                                 \\
    $\eta_1$                         & Upper bound for $\max_{x \in [a,b]} |\hat{r}^{(1)}_p(x)|$, given by $2\left(\frac{\sqrt{b}-\sqrt{a}}{\sqrt{b}+\sqrt{a}}\right)^p$.                        \\
    $\eta_2$                         & Upper bound for $\max_{x \in [a,b]} |\hat{r}^{(2)}_p(x)|$, expressed as $\left(\frac{b}{a}-1\right)^p$ when $p=N_{\text{tail}}$.                          \\
    $\bar{m}$                        & Total degree of the composite residual polynomial $r_{\bar{m}}$, formed as the product $\hat{r}^{(i)}_p(x)\hat{r}_{\bar{m}-p}(x)$.                        \\
    $P$                              & Performance ratio defined as $P = m_1/m_2$.                                                                                                               \\
    $P_{\text{uniform}}$             & Performance ratio for uniform eigenspectrum case.                                                                                                         \\
    $q(\kappa_l, \kappa_r)$          & Function appearing in uniform spectrum performance analysis.                                                                                              \\
    $T_{\kappa}(\kappa_l, \kappa_r)$ & Threshold function for determining when $m_2 < m_1$.                                                                                                      \\
    $W_{-1}(x)$                      & Lambert W function (principal branch $-1$).                                                                                                               \\
    $z^{(i,j)}_1, z^{(j)}_2$         & Chebyshev coordinates in the frame of reference of the $i^{\text{th}}$ cluster.                                                                           \\
    $\zeta^{(i,j)}_1, \zeta^{(j)}_2$ & Transformed Chebyshev coordinates for multi-cluster analysis.                                                                                             \\
    $k$                              & Positive integer parameter in the Chebyshev inequality, corresponding to the degree in the bound estimate.                                                \\
    $p_i$                            & Chebyshev degree associated with the $i^\text{th}$ eigenvalue cluster, determining the contraction factor of that cluster's contribution to the residual. \\
    $\kappa_i$                       & Condition number of the $i^\text{th}$ eigenvalue cluster, defined as the ratio of the largest to smallest eigenvalue within the cluster.                  \\
    $f_i$                            & Convergence factor (or spectral measure) for the $i^\text{th}$ cluster.                                                                                   \\
    $\Lambda_t$                      & Set of all eigenvalues residing in tail clusters.                                                                                                         \\
    $r_t(x)$                         & Residual polynomial for tail clusters: $r_t(x) = \prod_{\lambda\in\Lambda_t} \left(1 - \frac{x}{\lambda}\right)$.                                         \\
    $m_{N_{\text{cluster}}}$         & Multi-cluster CG iteration bound.                                                                                                                         \\
    $m_{N_{\text{tail-cluster}}}$    & Multi-tail-cluster CG iteration bound.                                                                                                                    \\
    $K^*$                            & Sorted partition indices for eigenspectrum clustering.                                                                                                    \\
    $k^*$                            & Split index that maximizes the ratio of consecutive eigenvalues.                                                                                          \\
    $I_t$                            & Set of tail cluster start indices.                                                                                                                        \\
\end{longtable}

% Symbols for Implementation
\begin{longtable}{c p{10cm}}
    \caption{Symbols related to implementation and numerical experiments.}\label{tab:implementation_symbols}                                                                      \\
    \hline
    \textbf{Symbol}                                    & \textbf{Description}                                                                                                     \\
    \hline
    \endfirsthead

    \hline
    \textbf{Symbol}                                    & \textbf{Description}                                                                                                     \\
    \hline
    \endhead

    \hline
    \endfoot

    \hline
    \endlastfoot

    $Q_h$                                              & Fine quadrilateral mesh with mesh size $h$.                                                                              \\
    $Q_H$                                              & Coarse quadrilateral mesh with mesh size $H$.                                                                            \\
    $h$                                                & Fine mesh size, where $h = H/2^r$.                                                                                       \\
    $H$                                                & Coarse mesh size.                                                                                                        \\
    $r$                                                & Refinement parameter such that $h = H/2^r$.                                                                              \\
    $\mathcal{Q}$                                      & Set of mesh pairs $\{(Q_h, Q_H)\}$ used in experiments.                                                                  \\
    $v^h_i$                                            & Internal fine mesh vertex.                                                                                               \\
    $q_j$                                              & Quadrilateral element in mesh.                                                                                           \\
    $\mathcal{C}_{\text{const}}$                       & Constant coefficient function $\mathcal{C}_{\text{const}} \equiv 1$.                                                     \\
    $\mathcal{C}_{\text{3layer, vert}}$                & High-contrast coefficient function with three-layer vertical structure.                                                  \\
    $\mathcal{C}_{\text{edge slabs, around vertices}}$ & High-contrast coefficient function with edge slabs around vertices.                                                      \\
    $\mathcal{M}^{-1}$                                 & Set of preconditioners $\{M^{-1}_{\text{2-OAS-GDSW}}, M^{-1}_{\text{2-OAS-RGDSW}}, M^{-1}_{\text{2-OAS-AMS}}\}$.         \\
    $M^{-1}_{\text{2-OAS-GDSW}}$                       & Two-level overlapping additive Schwarz preconditioner with GDSW coarse space.                                            \\
    $M^{-1}_{\text{2-OAS-RGDSW}}$                      & Two-level overlapping additive Schwarz preconditioner with RGDSW coarse space.                                           \\
    $M^{-1}_{\text{2-OAS-AMS}}$                        & Two-level overlapping additive Schwarz preconditioner with AMS coarse space.                                             \\
    $N_{\text{iter}}$                                  & Number of iterations for early estimation experiments.                                                                   \\
    $f_{\text{iter}}$                                  & Fraction parameter for determining $N_{\text{iter}}$.                                                                    \\
    $N_{\text{update}}$                                & Update frequency for eigenvalue convergence detection.                                                                   \\
    $\tau_{\text{extremal}}$                           & Tolerance parameter for extremal eigenvalue convergence.                                                                 \\
    $m_{\text{estimate}}$                              & Heuristic estimate defined as $m_{\text{estimate}} = \frac{1}{2}(m_{N_{\text{cluster}}} + m_{N_{\text{tail-cluster}}})$. \\
\end{longtable}

\section*{Abbreviations}

% Abbreviations
\begin{longtable}{c p{10cm}}
    \caption{List of abbreviations and their full meanings.}\label{tab:abbreviations} \\
    \hline
    \textbf{Abbreviation} & \textbf{Full Meaning}                                     \\
    \hline
    \endfirsthead

    \hline
    \textbf{Abbreviation} & \textbf{Full Meaning}                                     \\
    \hline
    \endhead

    \hline
    \endfoot

    \hline
    \endlastfoot

    CG                    & Conjugate gradient                                        \\
    PCG                   & Preconditioned conjugate gradient                         \\
    SPD                   & Symmetric positive definite                               \\
    FEM                   & Finite element method                                     \\
    ASM                   & Additive Schwarz method                                   \\
    RAS                   & Restrictive additive Schwarz                              \\
    ORAS                  & Optimized restrictive additive Schwarz                    \\
    2-OAS                 & Two-level overlapping additive Schwarz                    \\
    MsFEM                 & Multiscale finite element method                          \\
    ACMS                  & Approximate component mode synthesis                      \\
    GDSW                  & Generalized Dryja-Smith-Widlund                           \\
    RGDSW                 & Robust generalized Dryja-Smith-Widlund                    \\
    AMS                   & Algebraic multiscale solver                               \\
    DtN                   & Dirichlet-to-Neumann                                      \\
    DBC                   & Dirichlet boundary condition                              \\
    NBC                   & Neumann boundary condition                                \\
    PDE                   & Partial differential equation                             \\
    DOF(s)                & Degree(s) of freedom                                      \\
    LU                    & Lower-Upper decomposition                                 \\
    GMRES                 & Generalized minimal residual method                       \\
\end{longtable}

\pagestyle{fancy}